%Využití konceptu BYOD a jeho zabezpečení v bankovním prostředí
Cílem této práce bylo navrhnout využití konceptu BYOD ve vybrané bankovní organizaci, a to především s ohledem na jeho zabezpečení. Všechny body zadání byly splněny.

%Definujte klíčové faktory a rizika využití BYOD v bankovním prostředí.
S přihlédnutím k aktuálním trendům v ICT byly definovány klíčové faktory ve využívání nefiremních zařízení. Je zřejmé, že v dnešní době má velká část populace k dispozici svá soukromá zařízení a vzhledem ke zvýšení osobního komfortu a efektivity by někteří zaměstnanci tato zařízení rádi využívali také k pracovním účelům. Dalším častým důvodem zavádění BYOD ve firmách bývá snaha o snížení nákladů na ICT. S příchodem nefiremních zařízení do firemních sítí se ale pojí rizika jakými jsou například snadnější úniky dat, bezpečnostní incidenty nebo komplikovanější správa ICT.


%Ve vybrané bankovní organizaci analyzujte požadavky na připojení vlastních zařízení a identifikujte hlavní hrozby související s nasazením BYOD.
Ve vybrané organizaci byly formou četných konzultací analyzovány požadavky na BYOD. Z hlediska mobilních telefonů a tabletů je požadován především přístup k emailu a kalendáři, případně k dokumentům. V případě notebooků se požadavky liší podle typů uživatele, jedná se o potřeby jako přístup k interním systémům a prostředím, přístup k dokumentům či možnost instalace vlastních aplikací. Požadavkem Banky je zajistit co nejvyšší bezpečnost s co možná nejnižšími náklady.


%Analyzujte stávající bezpečností procesy v připojování nefiremních zařízení do její vnitřní sítě. 
Nevyhnutelné je využívání nefiremních zařízení pracovníky, kteří nejsou zaměstnanci Banky, ale pracují pro ni na živnostenský list nebo zprostředkovaně. Tato zařízení jsou připojována na základě bezpečnostních výjimek. Jelikož tato zařízení nejsou zkoumanou společností spravována, vyplývá ze stávající praxe bezpečnostní riziko. Existující procesy a hrozby byly analyzovány a jako nejnaléhavější potencionální hrozby byly vyhodnoceny následující: možné porušení bezpečnostních politik, možné zavedení škodlivého software do firemního prostředí či odcizení nebo poškození aktiv společnosti.


%Analyzujte dostupná existující řešení BYOD, zejména z pohledu bezpečnosti.
Předmětem analýzy byly také různé přístupy k řešení BYOD. Bylo zjištěno, že dostupná řešení je vhodné rozdělit na řešení pro notebooky a na řešení pro chytré telefony a tablety. S použitím studií od renomovaných analytických společností byli vyhodnoceni nejvýznamnější dodavatelé softwaru.


%Vyberte nejvhodnější variantu na trhu dostupného řešení a konfrontujte ji s praxí ve vybrané organizaci.
Na základě potřeb vybrané společnosti byla jako řešení pro notebooky zvolena virtualizace, a to specificky centralizovaná. Produktem, který nejlépe splňoval požadavky, se ukázal být Horizon Flex od společnosti VMware. Díky využití tohoto produktu je možné oddělit pracovní a soukromý operační systém. Pracovní operační systém lze plně spravovat a vynutit tak bezpečnostní politiky. Díky tomu jsou potlačeny nejzávažnější hrozby plynoucí ze stávajícího způsobu připojování.

Jako řešení pro mobilní telefony byl navržen Enterprise Mobility Suite od firmy BlackBerry, neboť splňuje požadavky uživatelů tím, že jim umožňuje přístup do firemního emailu,
kalendáře a dalších potřebných služeb, zároveň odděluje pracovní a soukromá
data a je hodnocen jako nebezpečnější řešení na trhu. Nezávisle na této práci byl během jejího dokončování započat projekt plošného nasazování řešení od BlackBerry do zkoumané organizace. Taktéž byla navržena další opatření nutná pro vznik uceleného BYOD programu.

%Konzultujte navrhované řešení se zástupci vybrané organizace a stanovte doporučení pro nasazení. 
Navrhované řešení bylo po celou dobu zpracování konzultováno se zástupci vybrané organizace. Bylo ohodnoceno jako vhodné a řešící stávající problémy v oblasti BYOD. V návaznosti na návrh byla stanovena doporučení pro nasazení zvolených produktů.

%Navrhněte nasazení řešení BYOD. 

%Zhodnoťte uskutečnitelnost řešení a analyzujte benefity a rizika spojená se zavedením navrženého konceptu.
Díky zmiňovanému řešení by se zvýšila bezpečnost BYOD ve zkoumané organizaci. Tato práce bude jedním ze vstupů pro budoucí projekt pro řešení BYOD ve společnosti.