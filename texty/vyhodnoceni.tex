%Zhodnoťte uskutečnitelnost řešení a analyzujte benefity a rizika spojená se zavedením navrženého konceptu.

Vzhledem k povaze problému, zjištěného analýzou uvnitř společnosti, bylo navrženo zvolit rozdílná řešení podle typu zařízení.

Bylo navrženo založit koncepci BYOD na tři základní pilíře a to:

\begin{itemize}
    \item Technické řešení pro notebooky
    \item Technické řešení pro mobilní zařízení
    \item Nastavení dalších opatření
\end{itemize}


\section{Vyhodnocení navrženého řešení}
\todo{Zhodnoťte uskutečnitelnost řešení a analyzujte benefity a rizika spojená se zavedením navrženého konceptu}
Z technického hlediska je návrh uskutečnitelný a po odladění technických detailů by z tohoto pohledu nic nebránilo jeho nasazení. Z pohledu obchodního má taktéž smysl, jelikož uskutečnění navržených opatření by se zvýšila bezpečnost informačních a komunikačních systémů zkoumané organizace, přičemž právě vnitřní bezpečnost je vzhledem k jejímu oboru podnikání důležitou součástí korporátní strategie. 

Hlavním benefitem řešení pro BYOD notebooky za použití distribuované virtualizace jsou nízké náklady na nasazení a provoz i vysoká flexibilita. Narozdíl od centralizované virtualizace s sebou nenese vysoké náklady na úložiště, servery a síťovou infrastrukturu. Stírá známé problémy centralizované virtualizace jako jsou nízká odezva, vysoká citlivost na kvalitu připojení či problémy s provozováním graficky náročných aplikací.

Zároveň však odděluje pracovní prostředí od operačního systému soukromého počítače a zajišťuje tak bezpečnost firemních dat. Velkým benefitem je možnost práce offline při zachování bezpečnosti díky užití virtualizace. Řeší problém s kontraktory a nekontrolovanými zařízeními ve vlastní síti.

Co se týče obecných výhod BYOD, pokud by do programu vstoupilo významné množství zaměstnanců, dá se předpokládat snížení nákladů na IT v rovině snížení nákladů na pořizování firemních zařízení. Dále se dá předpokládat zvýšení spokojenosti u zaměstnanců, kteří by vstoupili do BYOD programu z důvodu nespokojenosti s firemním zařízením. Tyto důvody by však byly pouze vedlejším efektem, hlavním důvodem pro zavedení BYOD programu je nastavení rámce pro existující nefiremní zařízení z důvodu bezpečnosti.

Největším rizikem nasazení tohoto řešení může být neochota uživatelů přistoupit na tento model. Přináší uživateli nutnost nainstalovat si na své zařízení klientský software a pracovat s ním. U kontraktorů, kteří jsou zaměstnanci partnerských dodavatelských společností není samozřejmostí povolení virtualizace na jejich pracovních zařízeních a je nutné tuto možnost pro vstup do BYOD programu zajistit.  Dále je třeba individuálně řešit licencování nejrůznějšího software, jelikož není možné v návrhu řešení BYOD programu obsháhnout všechny možné kombinace potřebného software a licenčních politik. Z hlediska bezpečnosti řešení odstraňuje aktuální bezpečností hrozby spojené s připojováním nefiremních zařízení. Bezpečnost se tak v tomto ohledu zvyšuje a nebyly identifikovány žádné přidané bezpečnostní hrozby. 

Hlavní benefit navrženého řešení pro BYOD mobilní telefony a tablety je vůbec možnost využití konceptu BYOD a mít tak možnost pracovat kdekoliv. To má potenciál zvýšení produktivity a spokojenosti zaměstnanců. Díky zvoleným licencím a typu nasazení uživatele nijak neomezuje v užívání jejich zařízení a nabízí tak vybalancování pracovního a soukromého života. Důvodem k navržení zavedení konkrétního produktu bylo vysoké zaměření na bezpečnost, které je v bankovním prostředí nejvyšší prioritou.

Rizikem může být nedůvěra zaměstnanců k firemnímu software, se schopností kontrolovat jejich soukromé zařízení. Reálný provoz také může ukázat u BYOD zařízení snížení výkonu či snížení výdrže na baterii v určitých konfiguracích. To by znamenalo negativní postoj zaměstnanců k BYOD programu a jeho možný neúspěch.

Procesní změny nutné k zavedení navrhovaného BYOD programu nejsou nikterak závažné a proto nejsou překážkou k realizaci projektu.


