Cílem této práce bylo zhodnotit koncept BYOD a jeho využití v bankovním prostředí. Analýza probíhala ve vybrané bankovní organizaci, požadavkem bylo najít vhodné řešení pro BYOD, s důrazem především na bezpečnost.

Tato práce probíhala v úzké spolupráci s vybranou bankovní organizaci, a to formou schůzek a konzultací zprostředkovaných oddělením IT Security. Práce odpovídá na otázky, co to BYOD je, co vede společnosti k úvahám o tomto konceptu a jaké jsou možnosti řešení.  Na základě analýzy vybrané bankovní organizace vybírá nejvhodnější řešení na aktuálním trhu a stanovuje doporučení pro jejich nasazení. Navržená řešení jsou vyhodnocena na základě benefitů a rizik spojených se zavedením a zpětné vazby od vybrané organizace.


Kapitola \ref{k1} je zaměřena na BYOD jako termín. Definuje jej s použitím několika zdrojů a uvádí jej do kontextu s aktuální situací v České republice i zahraničí. Zmiňuje důvody, proč je vhodné, aby se firmy problematikou zabývaly. Dále podrobněji popisuje obecně známe benefity a hrozby související se zavedením konceptu do firem. Závěr kapitoly se zabývá podněty nutnými ke zvážení z hlediska firemních politik, právního prostředí a především aktuální situace v oblasti kybernetické bezpečnosti.  



Kapitola \ref{k2} seznamuje čtenáře s výsledky analýzy vybrané bankovní organizace. Nejdříve je organizace krátce představena a jsou vyjmenována některá specifika tohoto typu organizací. Dále je podrobně analyzován stávající proces připojování nefiremních zařízení do firemní sítě včetně použitých technických prostředků a tento proces je dále podroben analýze souvisejících hrozeb. V neposlední řadě jsou analyzovány známé požadavky na BYOD v dané organizaci a typické skupiny uživatelů a jejich potřeb. Závěr kapitoly se zabývá projekty dotýkajícími se problematiky BYOD, které byly v organizaci uskutečněny již dříve.

Kapitola \ref{k3} se zaměřuje na analýzu existujících řešení na trhu. V první části jsou definovány různé pohledy na BYOD, a to na základě vlastnictví zařízení, typů zařízení a způsobu přístupu do datové sítě. Dále je čtenář seznámen se známými technickými řešeními pro BYOD. Na základě vlastností známých řešení je odděleně zvolen vhodný přístup řešení pro notebooky a mobilní zařízení. Pro tyto přístupy jsou vyhodnoceni nejvýznamnější poskytovatelé. 


V kapitole \ref{k4} je podrobně popsán návrh řešení pro BYOD. Volba produktů pro uskutečnění řešení je odůvodněna a taktéž je popsána jejich funkcionalita. Závěr kapitoly se věnuje nasazení vybraného řešení jak po technické stránce, tak po stránce formální. 

Poslední kapitola vyhodnocuje uskutečnitelnost navrženého řešení a analyzuje benefity a rizika spojená se zavedením navrženého konceptu.