\section{Cíl práce}
Cílem této práce bylo zhodnotit koncept BYOD a jeho využití v bankovním prostředí. Analýza probíhala ve vybrané bankovní organizaci, požadavkem bylo najít vhodné řešení pro BYOD a to především s důrazem na bezpečnost.

Tato práce probíhala v úzké spolupráci s vybranou bankovní organizaci a to formou schůzek a konzultací zprostředkovaných oddělením IT Security. Práce odpovídá na otázky co to BYOD je, co vede společnosti k úvahám o tomto konceptu a jaké jsou možnosti řešení.  Na základě analýzy vybrané bankovní organizace vybírá nejvhodnější řešení na aktuálním trhu a stanovuje doporučení pro jejich nasazení ve vybrané organizaci. Navržené řešení je vyhodnoceno na základě benefitů a rizik spojených se zavedením a zpětné vazby od vybrané organizace.

%%% COPY PASTE %%%%
\ref{k1}\todo{odcopypastovat}
Tato kapitola je zaměřena na BYOD jako termín. Definuje jej s použitím několika zdrojů a uvádí jej do kontextu s aktuální situací v České republice i zahraničí. Zmiňuje důvody, proč je vhodné, aby se firmy problematikou zabývaly. Dále rozvádí obecně známe benefity a hrozby zavedení konceptu do firem. Závěr kapitoly se zabývá podněty nutnými ke zvážení z hlediska firemních politik, právního prostředí a především aktuální situace z hlediska kybernetické bezpečnosti.  

%%% COPY PASTE %%%%
\ref{k2}\todo{odcopypastovat}
Tato kapitola seznamuje čtenáře s výsledky analýzy vybrané bankovní organizace. Nejdříve je organizace krátce představena a jsou vyjmenována některá specifika tohoto typu organizací. Dále je podrobně analyzován stávající proces připojování nefiremních zařízení do firemní sítě včetně použitých technických prostředků a tento proces je dále podroben analýze souvisejících hrozeb. V neposlední řadě jsou analyzovány známé požadavky na BYOD dané organizaci a typické skupiny uživatelů a jejich potřeb. Závěr kapitoly se zabývá dříve uskutečněnými projekty v organizaci dotýkající se problematiky BYOD.

