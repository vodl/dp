% options:
% thesis=B bachelor's thesis
% thesis=M master's thesis
% czech thesis in Czech language
% slovak thesis in Slovak language
% english thesis in English language
% hidelinks remove colour boxes around hyperlinks

\documentclass[thesis=M,czech]{FITthesis}[2012/06/26]

\usepackage[utf8]{inputenc} % LaTeX source encoded as UTF-8

\usepackage{graphicx} %graphics files inclusion
% \usepackage{amsmath} %advanced maths
% \usepackage{amssymb} %additional math symbols

\usepackage{dirtree} %directory tree visualisation

%Moje packages%%%%%%%%%%%%%%%%%%%%%%%%%%%%%%%%%%%%%
\usepackage{todonotes}
\usepackage{multirow}
\usepackage{hyperref}
\usepackage{enumitem}

% % list of acronyms
\usepackage[acronym,nonumberlist,toc,numberedsection=autolabel]{glossaries}
\iflanguage{czech}{\renewcommand*{\acronymname}{Seznam pou{\v z}it{\' y}ch zkratek}}{}
\makeglossaries

\newcommand{\tg}{\mathop{\mathrm{tg}}} %cesky tangens
\newcommand{\cotg}{\mathop{\mathrm{cotg}}} %cesky cotangens

% % % % % % % % % % % % % % % % % % % % % % % % % % % % % % 
% ODTUD DAL VSE ZMENTE
% % % % % % % % % % % % % % % % % % % % % % % % % % % % % % 

\department{Katedra softwarového inženýrství}
\title{Využití konceptu BYOD a jeho zabezpečení v bankovním prostředí}
\authorGN{Vojtěch} %(křestní) jméno (jména) autora
\authorFN{Dlápal} %příjmení autora
\authorWithDegrees{Bc. Vojtěch Dlápal} %jméno autora včetně současných akademických titulů
\supervisor{Ing. Pavel Krejčí}
\acknowledgements{Doplňte, máte-li komu a za co děkovat. V~opačném případě úplně odstraňte tento příkaz.}
\abstractCS{V~několika větách shrňte obsah a přínos této práce v~češtině. Po přečtení abstraktu by se čtenář měl mít čtenář dost informací pro rozhodnutí, zda chce Vaši práci číst.}
\abstractEN{Sem doplňte ekvivalent abstraktu Vaší práce v~angličtině.}
\placeForDeclarationOfAuthenticity{V~Praze}
\declarationOfAuthenticityOption{4} %volba Prohlášení (číslo 1-6)
\keywordsCS{Nahraďte seznamem klíčových slov v češtině oddělených čárkou.}
\keywordsEN{Nahraďte seznamem klíčových slov v angličtině oddělených čárkou.}

\begin{document}

\todo{doplnit podekovani}
\todo{napsat abstrakt}
\todo{prelozit abstrakt}
\todo{klicova slova}

% \newacronym{CVUT}{{\v C}VUT}{{\v C}esk{\' e} vysok{\' e} u{\v c}en{\' i} technick{\' e} v Praze}
% \newacronym{FIT}{FIT}{Fakulta informa{\v c}n{\' i}ch technologi{\' i}}


%%%%%%%%%%%%%%%%%%%%%%%%%%%%%%%%%%%%%%%%%%%%%%%%%%%%%%%%%%%%%%%%%%%%%%%%%%%%%%%%%%%
%%%%%%%%%%%%%%%%%%%%%%   ZADANI    %%%%%%%%%%%%%%%%%%%%%%%%%%%%%%%%%%%%%%%%%%%%%%%%
%%%%%%%%%%%%%%%%%%%%%%%%%%%%%%%%%%%%%%%%%%%%%%%%%%%%%%%%%%%%%%%%%%%%%%%%%%%%%%%%%%%

%Definujte klíčové faktory a rizika využití BYOD v bankovním prostředí.

%Analyzujte dostupná existující řešení BYOD, zejména z pohledu bezpečnosti.

%Ve vybrané bankovní organizaci analyzujte požadavky na připojení vlastních zařízení a identifikujte hlavní hrozby související s nasazením BYOD.

%Analyzujte stávající bezpečností procesy v připojování nefiremních zařízení do její vnitřní sítě.

%Vyberte nejvhodnější variantu na trhu dostupného řešení a konfrontujte ji s praxí ve vybrané organizaci. 

%Konzultujte navrhované řešení se zástupci vybrané organizace a stanovte doporučení pro nasazení. 

%Navrhněte nasazení řešení BYOD. 

%Zhodnoťte uskutečnitelnost řešení a analyzujte benefity a rizika spojená se zavedením navrženého konceptu.

\listoftodos[Co je treba udelat]
%%%%%%%%%%%%%%%%%%%%%%%%%%%%%%%%%%%%%%%%%%%%%%%%%%%%%%%%%%%%%%%%%%%%%%%%%%%%%%%%%%%
%%%%%%%%%%%%%%%%%%%%%%%%%%%%%%%%%%%%%%%%%%%%%%%%%%%%%%%%%%%%%%%%%%%%%%%%%%%%%%%%%%%
%%%%%%%%%%%%%%%%%%%%%%%%%%%%%%%%%%%%%%%%%%%%%%%%%%%%%%%%%%%%%%%%%%%%%%%%%%%%%%%%%%%
% --------------->>>> TEXT PRACE <<<<<<<------------------------------%
%%%%%%%%%%%%%%%%%%%%%%%%%%%%%%%%%%%%%%%%%%%%%%%%%%%%%%%%%%%%%%%%%%%%%%%

\begin{introduction}
	%sem napište úvod Vaší práce
	Cílem této práce bylo zhodnotit koncept BYOD a jeho využití v bankovním prostředí. Analýza probíhala ve vybrané bankovní organizaci, požadavkem bylo najít vhodné řešení pro BYOD, s důrazem především na bezpečnost.

Tato práce probíhala v úzké spolupráci s vybranou bankovní organizaci, a to formou schůzek a konzultací zprostředkovaných oddělením IT Security. Práce odpovídá na otázky, co to BYOD je, co vede společnosti k úvahám o tomto konceptu a jaké jsou možnosti řešení.  Na základě analýzy vybrané bankovní organizace vybírá nejvhodnější řešení na aktuálním trhu a stanovuje doporučení pro jejich nasazení. Navržená řešení jsou vyhodnocena na základě benefitů a rizik spojených se zavedením a zpětné vazby od vybrané organizace.


Kapitola \ref{k1} je zaměřena na BYOD jako termín. Definuje jej s použitím několika zdrojů a uvádí jej do kontextu s aktuální situací v České republice i zahraničí. Zmiňuje důvody, proč je vhodné, aby se firmy problematikou zabývaly. Dále podrobněji popisuje obecně známe benefity a hrozby související se zavedením konceptu do firem. Závěr kapitoly se zabývá podněty nutnými ke zvážení z hlediska firemních politik, právního prostředí a především aktuální situace v oblasti kybernetické bezpečnosti.  



Kapitola \ref{k2} seznamuje čtenáře s výsledky analýzy vybrané bankovní organizace. Nejdříve je organizace krátce představena a jsou vyjmenována některá specifika tohoto typu organizací. Dále je podrobně analyzován stávající proces připojování nefiremních zařízení do firemní sítě včetně použitých technických prostředků a tento proces je dále podroben analýze souvisejících hrozeb. V neposlední řadě jsou analyzovány známé požadavky na BYOD v dané organizaci a typické skupiny uživatelů a jejich potřeb. Závěr kapitoly se zabývá projekty dotýkajícími se problematiky BYOD, které byly v organizaci uskutečněny již dříve.

Kapitola \ref{k3} se zaměřuje na analýzu existujících řešení na trhu. V první části jsou definovány různé pohledy na BYOD, a to na základě vlastnictví zařízení, typů zařízení a způsobu přístupu do datové sítě. Dále je čtenář seznámen se známými technickými řešeními pro BYOD. Na základě vlastností známých řešení je odděleně zvolen vhodný přístup řešení pro notebooky a mobilní zařízení. Pro tyto přístupy jsou vyhodnoceni nejvýznamnější poskytovatelé. 


V kapitole \ref{k4} je podrobně popsán návrh řešení pro BYOD. Volba produktů pro uskutečnění řešení je odůvodněna a taktéž je popsána jejich funkcionalita. Závěr kapitoly se věnuje nasazení vybraného řešení jak po technické stránce, tak po stránce formální. 

Poslední kapitola vyhodnocuje uskutečnitelnost navrženého řešení a analyzuje benefity a rizika spojená se zavedením navrženého konceptu.\todo{napsat úvod}
\end{introduction}

\chapter{Charakteristika BYOD}\label{k1}
%Definujte klíčové faktory a rizika využití BYOD v bankovním prostředí.
\section{Charakteristika BYOD}
\todo{pouzit zdroj consumeration neco}
\todo{pouzit zdroj ZDnet: http://www.zdnet.com/article/enterprise-mobility-byod-emm-and-new-security-approaches/}

\section{Trendy v BYOD} \missingfigure{Obrazek z google trends}\todo{nejake statistiky a magic quadranty}


\section{Specifika bankovního prostředí}


\section{BYOD z hlediska bezpečnosti}

\chapter{Analýza prostředí a požadavků}\label{k2}
 %Ve vybrané bankovní organizaci analyzujte požadavky na připojení vlastních zařízení a identifikujte hlavní hrozby související s nasazením BYOD.


\section{Aktuální stav připojování soukromých zařízení}\todo{Učesat a zformulovat}

Komerční banka používá vnitřní informační systém is.kb.cz. Jeho součástí je i service desk pro přidávání uživatelů. Správa servicedesku je outsourcována na firmu HP. Pro připojení vlastního zařízení do sítě KB je nutné zadat do systému výjímku. Pracovní postup pro přidání výjimky je následující: přidá se požadavek, dále probíhá schvalování, požadavek je vyhodnocen IT security a po schválení je zadání výjimky outsourcováno na externí firmu (HP). 


Pro správu zařízení KB používá nástroj MAB Keeper od firmy AleFIT. Ta jej definuje jako: \textit{Aplikace slouží ke správě MAC adres zařízení, která jsou v autentizačním systému použita pro autentizaci, ale nejsou kompatibilní se standardem 802.1x, nebo správě zařízení, u nichž se MAC adresa využívá jako náhradní způsob autentizace. AleFIT MAB Keeper také umožňuje díky několika modulům kontrolovat a časově omezit přístup kontraktorů, konzultantů i BYOD zařízení do firemní sítě, stejně jako využít workflow pro realizaci re-image stanic.} \todo{citace https://www.alef.com/alefnula/alefit-mab-keeper-a-alefit-office-locator.c-269.html} Je to vlastně nadstavba nad systémem Cisco Identity Service Engine (ISE).


Systém ISE řídí nasměrování zařízení do patřičné VLAN na základě adresy MAC. Díky tomu řídí práva zařízení. Jedná se tedy o správu připojených zařízení na úrovni topologie sítě. Výjimky je možné najít v MAPkeeperu pod Approved devices. MapKeeper slouží pro distribuci správy. Používá se Mac address bypass i další metody co mohou selhat. Pokud dojde při ověření MAC adresy k chybě, použije se  802.1.X  


Jelikož nefiremní zařízení nemají autorizační certifikát, používá se autorizační funkce. Uživatel s vlastním zařízením s uznanou výjimkou se připojuje do stejné VLAN jako KB zařízení. \todo{a to je fail}


Pro připojení do sítě se používají CISCO ISE a VPN. Dochází k různým kontrolám. Pokročilejší kontroly umožňuje cisco ISE plus, kdy na jednoho uživatele je zapotřebí jedna licence. KB momentálně vlastní balík 500 licencí. Cena byla zhruba pul milionu korun. Bude docházet k výměně zařízení ASA a licence budou převedeny. Při použití principů NAC (Network Access Control) dochází k problému, kdy při změně zásuvky, může být uživateli přidělena jiná VLAN než je uživateli běžně přidělována při použití zásuvky na jeho obvyklém pracovním místě.


Z hlediska použití bezdrátových sítí WIFI existují admin sítě, ke kterým je možné přistupovat pouze přes terminál \todo{WTF}. Jedná se o standardní datovou síť. Pro přístup je nutný certifikát, a to především z důvodu fyzické dostupnosti signálu sítě i mimo objekty KB. Není v plánu další rozšiřování datových WIFI sítí.  

Dále existuje WIFI síť pro hosty pouze s přístupem na internet. Probíhá na ní URL filtrace.

\section{Cisco ISE}
Společnost Gartner \todo{cituj https://www.gartner.com/doc/reprints?id=1-3I8W2V2\&ct=160922\&st=sb} popisuje Cisco ISE jako jako tachnologii založenou na protokolu RADIUS. Pokročilé funkce NAC potřebují užitích dalších komponent jako třeba TrustSec Security Group Tag. S použitím device profiling and feed service umožňuje analyzovat provoz a vytvářet reporty o připojených zařízeních.

Balík Cisco AnyConnect sjednocuje další funkce jako jsou VPN, NetFlow nebo ochrana proti škodlivému software. V ISE verze 2.0 je zabudována podpora pro certifikáty, Active Directory či TACACS+.

\section{Analýza aktuálních bezpečnostních rizik}
Aktuální bezpečnostní rizika jsou vysoká, jelikož dodržování bezpečnosti je jen na dobré slovo. \todo{rozvést, popsat, doplnit byrokracii, doplnit proces, papiry nda, papiry co se musi podepsat... }
\todo{popsat moznost zaneseni viru atd.}

\section{Analýza požadavků na připojení vlastních zařízení} \todo{Udělat seznam požadavků na BYOD}

\section{Analýza hrozeb souvisejících s nasazením BYOD}

\section{Analýza požadavků}
Projekt na vyhodnocení konceptu BYOD byl v Komerční bance započat již v roce 2013. Byla snaha o vyhodnocení rámce pro konkrétní potřeby, scénáře a služby, dále měly být nastaveny předpokládané výstupy a definování možného dosažení řešení. Již v roce 2013 byl citelný příklon uživatelů ke konzumerizaci informačních technologií a prorůstání nonPC zařízení do firemního prostředí.

Mobilní připojení k internetu se stalo standardem i pro běžné uživatele a ti tak byli neustále připojeni se svými osobními zařízeními k internetu. Dále byla citelná osobní potřeba zaměstnanců používat svá osobní zařízení i během pracovní doby. Byla nastolena možnost dát zaměstnancům možnost přístupu k emailu i z nefiremních zařízení, což by mohlo zvýšit pracovní efektivitu při minimálních dodatečných výdajích. Pro mnohé pracovní pozice by též zavedení BYOD programu mohlo umožnit flexibilnější pracovní styl. To se týká i snadnějšího přístupu ke klientům, a tedy umožnění pracovníkům z prodeje být blíže klientovi. Díky flexibilnějšímu přístupu by mohlo být možné lépe uplatňovat techniky křížného prodeje\footnote{Křížný prodej někdy též křížový prodej (anglicky cross-selling) je obchodní taktika navyšování prodeje, jejímž cílem je prodat více doporučením souvisejícího zboží nebo služeb.}.\todo{citace https://managementmania.com/cs/krizovy-prodej-cross-selling} Bezprostřední přístup k informacím by umožnil rychlejší reakci obchodníků a konkurenční výhodu. 

Je třeba nalézt takové řešení, které bude zodpovědné z hlediska nákladů. Zároveň je potřeba, aby řešení mělo kladné přijetí od potencionálních uživatelů, tak aby byli ochotni jej využívat a náklady na zavedení nebyly vynaloženy zbytečně.  

Je tedy možné rozlišit jak technologicko-sociální důvody k zavedení řešení pro BYOD tak také důvody businessové. Je zřejmé, že vzhledem k nastoleným trendům je nutné nastolit firemní strategii pro BYOD. Pokud by se řešení nenašlo, nefiremní zařízení se přesto budou rozšiřovat, ovšem nebudou pod kontrolou firemního IT oddělení. To v důsledku znamená, že není možné kontrolovat jak rizika tak náklady s tímto spojené. 


\subsection{Identifikované potřeby businessu}
\begin{itemize}
    \item Tablety pro vrcholový management
    \item Vlastní notebooky
    \item Vlastní počítače Apple
    \item Firemní notebooky externích konzultantů
    \item Přístup k dokumentům ze soukromých tabletů
    \item Přístup k emailu či kalendáři z osobního chytrého telefonu
\end{itemize}

\subsection{Identifikované služby}
Z hlediska souvisejících služeb poskytovaných IT byly identifikovány služby typu \textbf{PIM}\footnote{Personal Information Management} neboli služby pro správu osobních informací, typicky se jedná o kalendář, email a další komunikační a organizační systémy. Prakticky se jedná o Oultook a Skype for Bussiness. Dále zprostředkovat služby pro \textbf{tvorbu a sdílení dokumentů}. Typicky se jedná o Microsoft Office či Atlassian Confluence. V neposlední řadě je nutný přístup k \textbf{business aplikacím}.

\subsection{Identifikované typy zařízení}
Co se týče různých typů zařízení, je třeba do BYOD programu zařadit firemní chytré telefony včetně zařízení BlackBerry a Tablety. Co se týče nefiremních či osobních zařízení je třeba zohlednit chytré telefony, tablety, notebooky či notebooky od firmy Apple.

\subsection{Identifikovaní uživatelé}
Jako uživatelé byli identifikování zaměstnancí KB, externí dodavatelé a kontraktoři a klienti.

\subsection{Způsoby připojení k síti}
Z hlediska připojení k síti byly identifikovány následující možnosti: připojení do sítě LAN, připojení do lokální WiFi a připojení skrze síť internet a mobilní připojení.

\subsection{Způsob podpory od IT oddělení}
Momentálně IT oddělení poskytuje end-to-end podporu. To znamená, že dodaní služeb je podporováno kompletně od zdroje, přes dodání po podporu koncových zařízení. Tento model není trvale udržitelný pro BYOD, kdy není možné podporovat všechna koncová zařízení a je tedy třeba zavést i model, kde je podporována pouze samotná služba.

\subsection{Identifikace potřeb specifického uživatele -- vývojáře}
Vývojáři patří mezi prioritní skupinu uživatelů, pro které je třeba připravit projekt BYOD. Je to především proto, že právě mezi vývojáři je velké množství kontraktorů, kteří si přinášejí své vlastní nefiremní zařízení. Vývojáři však májí vyšší požadavky než běžní uživatelé. Konzultací se zástupci vývojářů v Komerční bance bylz identifikovány následující potřeby. 

Vývojář potřebuje mít na zařízení na kterém vyvíjí administrátorská oprávnění. Je to především z důvodu instalace pomocných nástrojů, tak z důvodu přístupu k některým systémovým funkcím operačního systému a to například pro potřeby testování. Dále má vývojář zvýšené nároky na výpočetní výkon stroje na kterém pracuje a to především z důvodu potřeby lokální kompilace zdrojových kódů. 

Vývojáři mají specifické požadavky na nainstalované aplikace. Každý potřebuje vývojové prostředí (KB nemá sjednoceno a tedy vývojáři mohou volit nástroj dle svého uvážení, například IntelliJ Idea). Dále jsou to nástroje pro vývoj databází, například Oracle SQL Developer. Dále je nutné přistupovat k dalším databázím. V prostředí KB se používají různé databáze (Oracle, MySQL, MS SQL). Mezi dalšími nezbytnými nástroji byl uveden SSH klient Putty.

Byla zmíněna potřeba přístupu k následujícím službám:
\begin{itemize}
    \item Přihlašování do domény
    \item Přístup k logům -- k centrálnímu systému logů na systému Logman
    \item Přístup k nástroji pro zpracování výstupních streamů Apache Kafka
    \item Přístup k verzovacímu systém GIT na platformě BitBucket
    \item Přístup k systému pro evidenci chyb Atlassian JIRA
    \item Přístum k nástroji pro dokumentaci Atlassian Confluence
    \item Přístup k nástroji pro automatizaci správy software Jenkins
    \item Přístup k testovacím prostředí
    \item Přístup k emailům 
    \item Přístup ke službě Skype for business
    \item Přístup k adresářové službě LDAP
    \item Přístup ke správě identit ITIM
\end{itemize}

Pokud se vývojář připojuje vzdáleně, klade důraz na přístup k verzovacíu systému GIT, přístup k testovacímu prostředí a přístup k logům.

\section{Aktuální stav podpory různých typů zařízení dle typu vlastnictví}


Nejvyšší prioritou je umožnit uživatelům se soukromými zařízeními plný a kontrolovaný přístup ke službám lokální sítě. Není nutné zajišťovat vzdálený přístup pro nefiremní zařízení, je však třeba podporovat vzdálený přístup k emailu.Pro mobilní telefony je třeba zajistit přístup k emailu, přístup k dokumentů a aplikacím zatím není vyžadován. Byl identifikován požadavek na firemní tablety od vrcholového managementu. Pro ty je třeba zajistit maximální přístup. Uživatelé soukromých tabletů požadují přístup k emailu a dokumentům.

\section{Dříve zvažované možnosti pro email}
V minulosti bylo zvažováno několik možností, jak zpřístupnit email a dokumenty na soukromých chytrých telefonech a tabletech.

Exchange ActiveSync snižuje riziko krádeže díky vynucení zadání PIN kódu a možnosti vzdáleného smazání. V jeho prospěch hrála relativně snadná a rychlá implementace. Řešení bylo zamítnuto protože nenabízelo zašifrování dat, což ohrožení dat například při jail-breaku u zařízení iPhone.

V rámci mateřské skupiny se používá řešení Good mail (nyní BlackBerry Work). Toto řešení nevyhovovalo požadavkům vzhledem k vysokým nákladům na implementaci a provoz. Odhad činil 30.000 EUR a 70 dnů lidské práce na přípravu infrastruktury a 152 EUR za licenci na uživatele na rok a dále 24EUR na uživatele a rok jako náklad na údržbu. Nicméně projekt pro toto řešení stále existuje.

Microsoft Outlook Web App umožňuje prohlížení příloh přímo na serveru a není tedy nutné lokální šifrování dat. Řešení je vhodné jak pro mobilní zařízení, tak tablety a nepřináší žádné dodatečné náklady, jelikož je již implementováno. Uživatelé však nehodnotí uživatelskou přívětivost tohoto řešení příliš kladně.

\section{Projekt VDI pro vývojáře a testery}\label{projektVDI}

V KB též existoval projekt, který se snažil zjisti možnost využití virtuálních strojů pro vývojáře a testery. Důvodem byla snaha získat řešení pro vlastní zařízení kontraktorů, které znamenají bezpečnostní riziko. Dále si KB od projektu slibovala nalezení řešení problému s využíváním několika zařízení vývojáři a to ať už z důvodů vzdálené podpory nebo zvláštních požadavků na výkon.

Test VDI se odehrál 21.11.2011, probíhal tři týdny a zúčastnilo se jej 5 vývojářů. Uživatelé po dobu testu prováděli veškeré své pracovní úkony ve virtuálním prostředí. Virtuální stroje běželi na serveru Proliant DL380 G5 s parametry: 4x (2CPUx2jádra) CPU 3000MHz, 24GB RAM a 1,3TB místa na disku. Parametry pro jednotlivé virtuální stroje byly ekvivalentí k PC dvoujádrovým procesorem a 2-4 GB RAM.

Účastníci testu ohodnotili uživatelský zážitek jako dostatečný pro běžné použití. Zaznamenali však nižší odezvu a občasné záseky. Byly identifikovány problémy s periferními zařízeními, například nefungovala čtečka na čipové karty. Odezva vývojářských nástrojů byla odhadem dvakrát pomalejší. Uživatelé hodnotili přechod k virtuálním strojům jako zhoršení uživatelského komfortu oproti fyzickým firemním PC. 

Test prokázal vysoké nároky na diskové úložiště a to především co se týče počtu požadavků na vstupně/výstupní operace. To znamená nutnost vysoké investice do kvalitního diskového úložiště. Zároveň je nutné zajistit kvalitní konektivitu. Proto bylo rozhodnuto, že VDI není vhodné pro interní vývojáře, protože zvyšuje náklady a nepřináší benefity.

Zároveň test doporučil ke zvážení zkoušený model VDI pro kontraktory a to pod podmínkou užití vlastního zařízení bez dalších nákladů pro KB, bez zajištění vysoké dostupnosti a omezení velikosti diskové kapacity pro virtuální stroje na ~70-80GB.






 
\chapter{Možné varianty řešení}\label{k3}
 %Analyzujte dostupná existující řešení BYOD, zejména z pohledu bezpečnosti.
Tato kapitola se zaměřuje na analýzu existujících řešení na trhu. V první části jsou definovány různé pohledy na BYOD na základě vlastnictví zařízení, typů zařízení a způsobu přístupu do datové sítě. Dále jsou definovány známá technická řešení pro BYOD. Na základě vlastností známých řešení je odděleně zvolen vhodný přístup řešení pro notebooky a mobilní zařízení. Pro tyto přístupy jsou vyhodnoceni nejvýznamnější poskytovatelé. 


 \section{Různé pohledy na BYOD}
 Na problematiku nefiremních zařízení je možné nahlížet z různých úhlů pohledu. V této sekci budou představeny různé možnosti dělení zařízení dle různých kategorií a taktéž budou představeny odpovídající řešení.
 
 \subsection{Rozdělení zařízení podle vlastnictví}
 Na základě toho, kdo je vlastníkem zařízení připojovaného k firemní síti, je určena míra kontroly zařízení firmou.
 
 \subsubsection{Firemní zařízení}
 Jedná se o zařízení, které nakupuje a zároveň spravuje firma. Je ve vlastnictví firmy a pod dohledem IT oddělení. Zařízení plně splňuje politiky firmy a je plně kontrolované.
 
 \subsubsection{Externí firemní}
 Jedná se o firemní zařízení pracovníka externí firmy, jedná se tedy o firemní zařízení, ale jiné firmy. Je tedy pod správou IT oddělení externí firmy.
 
 Zařízení splňuje bezpečnostní politiky externí firmy a je kontrolované externí firmou. Není možné zařízení kontrolovat, je však možné vynutit potřebné bezpečnostní politiky smluvním vztahem s externí firmou.
 
 \subsubsection{Externí soukromé}
 Zařízení externího pracovníka, které není kontrolované firemní politikou externí firmy. Není možné jej kontrolovat a je obtížné vynucovat bezpečnostní politiky.
 
 \subsubsection{Zaměstnanec se soukromým zařízením}
 Vlastní zařízení zaměstnanců. Není kontrolované a může představovat bezpečnostní hrozbu.
 
 \subsection{Rozdělení podle typu zařízení}
 \subsubsection{PC}
 V užším slova smyslu se jedná o osobní počítače s operačním systémem Windows od firmy Microsoft, viz \cite{Intel_Mac_PC}. Windows je aktuálně nejrozšířenější operační systém pro korporátní zařízení. Systém je určený pro zařízení postavená na architektuře x86. Typicky se jedná o stolní počítače a notebooky. Podle statistiky StatCounter \cite{Statcounter1} měl operační systém Windows v únoru 2017 90 procentní podíl na trhu operačních systémů pro desktopy podle počtu přístupů na web. Podle celosvětových statistik přístupů na web v souhrnu všech typů zařízení podle StatCounter má však Windows pouze 38,6 procenta přístupů a mezi běžnými uživateli je zřejmá tendence v upřednostňování jiných zařízení na úkor PC \cite{HNAndroid}.
 
\begin{figure}[h!]
%\includegraphics[width=13cm]{img/StatCounter_Desktop}
\centering
\includegraphics[width=10cm]{img/1_Desktopy_CZ}
\caption{Statistika podílu operačních systémů pro desktopy v České republice podle přístupů na web. Převzato z \cite{Statcounter1}.} 
\centering
\end{figure}
 
 
 
 \subsubsection{Mac}
 Počítač s operačním systémem MAC OS \cite{AppleMacOS}. Jedná se o proprietární operační systém pro počítače firmy Apple. Podle StatCounter je jeho podíl na trhu mezi desktopovými operačními systémy podle počtu přístupů na web v České republice necelých šest procent. Populární je zejména ve Spojených státech, kde se jeho podíl mezi desktopovými operačními systémy podle metodiky StatCounteru pohybuje okolo dvaceti procent.
 
\begin{figure}[h!]
%\includegraphics[width=13cm]{img/StatCounter_Destop_USA}
\centering
\includegraphics[width=10cm]{img/2_Desktopy_US}
\caption{Statistika podílu operačních systémů pro desktopy v USA podle přístupů na web. Převzato z \cite{Statcounter1}} 
\centering
\end{figure}%\todo{citace}
 
 
 \subsubsection{Chytrý telefon či tablet}
 Podle společnosti Gartner \cite{GartnerSmartphone} je chytrý telefon definován jako mobilní komunikační zařízení používající identifikovatelný otevřený operační systém. Tento systém je podporován aplikacemi třetích stran od komunity vývojářů. Aplikace třetích stran mohou být instalovány nebo odstraněny a mohou být vytvořeny přímo pro operační systém zařízení a aplikační programové rozhraní, případně pro oddělenou vrstvu jakou může být například Java. Operační systém musí podporovat multitaskingové prostředí a uživatelské rozhraní, které dokáže obsloužit více aplikací najednou. Například zobrazení emailu během přehrávání hudby.
 
  \begin{figure}[h!]
%\includegraphics[width=13cm]{img/StatCounter_MobileBar}
\centering
\includegraphics[width=10cm]{img/3_Mobil_CZ}
\caption{Statistika podílu operačních systémů pro mobilní zařízení v České republice podle přístupů na web. Převzato z \cite{Statcounter1}.} 
\centering
\end{figure}
 
 Obecněji se jedná o mobilní zařízení s možností instalace aplikací. V současné době jsou nejpopulárnější zařízení s operačním systémem Android od firmy Google a iOS od firmy Apple. V České republice je podle metodiky měření společnosti StatCounter pro únor 2017 nejpopulárnější operační systém Android s podílem 68 procent. Operační systém iOS je na drůhém místě s podílem 26 procent. Více než jednoprocentní podíl má již pouze Windows s necelými čtyřmi procenty \cite{HNAndroid}. 
 

 
Celosvětově je zřejmá rostoucí obliba mobilních zařízení mezi uživateli, a to především na úkor klasických PC s operačním systémem Windows. Vzhledem k postupné změně návyků uživatelů je vyžadováno, aby firemní prostředí na tento trend vhodně reagovalo. 


\begin{figure}[h!]
%\includegraphics[width=13cm]{img/StatCounter_VyvojVse}
\centering
\includegraphics[width=13cm]{img/4_vyvoj_vse_CZ}
\caption{Statistika vývoje podílů všech operačních systémů v České republice podle přístupů na web. Převzato z \cite{Statcounter1}.} 
\centering
\end{figure}

\begin{figure}[h!]
%\includegraphics[width=13cm]{img/StatCounter_All_Worldwide}
\centering
\includegraphics[width=13cm]{img/5_vyvoj_vse_global}
\caption{Statistika vývoje podílů všech operačních systémů celosvětově podle přístupů na web. Převzato z \cite{Statcounter2}.} 
\centering
\end{figure}
 
U těchto zařízení se nepředpokládá nutnost přístupu k podnikovým aplikacím, ale je vyžadován okamžitý přístup k emailům, kontaktům či dokumentům, a to nezávisle na místě použití.

Dle reportu od společnosti Nokia Thread Intelligence Lab \cite{Nokia2, Nokia1}, který monitoroval aktivitu malwaru v sítích mobilních operátorů mezi lety 2012 a 2015 měly 60 \% veškeré aktivity malwaru v mobilních sítích na svědomí chytré telefony, zbytek šel na vrub Windows PC.

V prosinci 2015 vykazovalo známky napadení škodlivým softwarem 0,3 \% všech chytrých telefonů. Nejvíce napadení zaznamenala zařízení s operačním systémem Android, ovšem na seznam s dvaceti nejčastěji se vyskytujícími druhy škodlivého software se dostali i dva zástupci pro operační systém iOS (XcodeGhost a Flexispy). V říjnu 2015 bylo 6 \% všech napadených zařízení značky iPhone.
 
 
 %%%%%%%%%%%%%%%%%%%%%%%%%%%%%%%%%%%%%%%%%%%%%%%%%%%%%%%%%%%%%%%%%%%%%%%%%%%%%%%%%%%%%%%%%%%%%%%%%%%%%%%%%%%%%%%%%%%%%%%%%%%%%%%%%%%
 \subsection{Rozdělení podle typu přístupu do datové sítě}
 \subsubsection{Ethernet}
 Jedná se o pevné připojení do sítě pomocí kabelu \cite{pcmagEthernet}. Je vhodné pro firemní počítače, není vhodné pro zařízení typu mobilní telefon či tablet.
 
 Je definované ve standardu IEEE 802.3.
 
 \subsubsection{WiFi}
 
 Bezdrátové připojení pomocí WiFi sítí. Jedná se o standardní technologii pro bezdrátové sítě WLAN \cite{pcmagWifi}. Tento typ připojení je vhodný pro přenosné počítače, mobilní telefony i tablety. WiFi je definováno standardem IEEE 802.11.
 
 \subsubsection{VPN}
 
 Virtual private network čili vzdálené připojení do firemní sítě. Hlavním smyslem VPN je vytvořit soukromou síť pomocí tunelování a nebo šifrování skrze veřejný internet tak, aby uživatelé mohli vzdáleně přistupovat ke službám dostupným pouze zevnitř sítě \cite{ciscoJournal}.
 
 
 %%%%%%%%%%%%%%%%%%%%%%%%%%%%%%%%%%%%%%%%%%%%%%%%%%%%%%%%%%%%%%%%%%%%
 
 
 
 \section{Známé způsoby řešení BYOD}%\todo{Zejmena z pohledu bezpecnosti}


\subsection{Virtualizace} 
Podle \cite{Shackleford} je virtualizace abstrakcí výpočetních zdrojů od fyzické hardwarové vrstvy. Virtuální stroj vystupuje jako samostatný výpočetní systém dostupný z jiného stroje. Díky virtualizaci je možné oddělit data a prostředí fyzického a virtuálních strojů. Jako hostitel je nazývána platforma na které běží hypervizor. Virtuální stroj je pak systém na kterém běží virtuální prostředí. Reprezentuje kompletní hardwarovou platformu. Virtuální stroje pak běží nad hypervizorem. 

Hypervizor je hlavní komponenta virtualizece. Rozeznáváme 2 druhy:

\subsubsection{Hypervizor typu 1}
Hypervizory prvního typu jsou nainstalovány přímo nad hardware. Mezi hardware a virtuálními stroji s operačními systémy tak není žádná další vrstva. Hypervizory typu 1 jsou zpravidla instalovány na servery ve výpočetních střediscích.

\begin{figure}[h!]
\centering
\includegraphics[width=10cm]{img/shackleford1}
\caption{Schéma virtualizace s hypervizorem typu 1. Převzato z \cite{Shackleford}.} 
\end{figure}

\subsubsection{Hypervizor typu 2}
Hypervizory druhého typu jsou aplikace instalované v rámci existujícího operačního systému. Mezi hardware a operačními systémy virtuálních strojů je tak navíc vrstva operačního systému hostitelského stroje. Hypervizory typu 2 jsou zpravidla instalovány na pracovní stanice.

\begin{figure}[h!]
\centering
\includegraphics[width=10cm]{img/shackleford2}
\caption{Schéma virtualizace s hypervizorem typu 2. Převzato z \cite{Shackleford}.} 
\end{figure}

\subsection{Hrozby spojené s virtualizací}
Hlavní výhodou virtualizace je oddělení jednotlivých prostředí. Samotnou virtualizací se však bezpečnost nezvyšuje a ve virtualizovaném prostředí existují stejné hrozby jako v prostředím fyzickém. Z hlediska BYOD je však právě oddělení prostředí tou zásadní vlastností, jelikož úroveň zabezpečení firemního virtuálního stroje lze odděleně spravovat.

Kniha \cite{Shackleford} identifikuje několik hrozeb spojených s provozem virtuálních strojů. Relevantní pro tuto práci jsou:

\begin{description}
  \item[Škodlivý software] Byly objeveny některé druhy škodlivého softwaru, které dokáží detekovat, že se nachází ve virtualizovaném prostředí. Díky tomu dokáží modifikovat svoje chování a lépe se tak maskovat.
  \item[Únik z virtuálního stroje] Podle \cite{Shackleford} zatím nebyl zaznamenán takový útok, kdy by kódu běžícímu uvnitř virtuálního stroje podařilo dostat ven a ohrozit tak hostitelský systém nebo jiný virtuální stroj. Koncept toho druhu útoků byl však několikrát dokázán v laboratorních podmínkách. Pro tyto koncepty je však nutné připravit jak software uvnitř hostitelského tak virtualizovaného systému, nebo využít některou z funkcí klienta pro sdílení mezi hostitelským a virtualizovaným systémem.
  \item[Další zranitelnosti] Chyby v softwaru pro virtualizaci mohou znamenat například možnost vzdáleného vyřazení stanice z provozu, spuštění škodlivého kódu nebo dalších útoků. Zranitelnosti jsou předmětem bezpečnostních záplat.
\end{description}

 \subsection{Centralizovaná virtualizace}
 Pod pojmem centralizovaná virtualizace se rozumí běh virtuálních strojů na serveru ve výpočetním středisku. Používají se tedy hypervizory typu 1. Klienti k těmto virtuálním strojů přistupují vzdáleně pomocí technologie VDI. Nejsou tedy spotřebovávány výpočetní prostředky klienta, ale je zapotřebí kvalitní konektivita do výpočetního střediska.
 
 Nejznámějšími zástupci centralizované virtualizace jsou: Microsoft remote desktop, VMWare Horizon, Citrix XenDesktop.
 
 \subsection{Distribuovaná virtualizace}
 Pojmem distribuovaná virtualizace se rozumí běh virtuálních strojů přímo na koncových stanicích uživatelů.  Používá se tedy hypervizor typu 2. Spotřebovávají se tedy výpočetní prostředky stroje klienta, není však obecně potřeba konektivita s vnějším světem.
 
 Nejznámějšími zástupci jsou: Oracle Virtualbox, VMWare Fusion, VMWare Workstation, VMWare Player, VMWare Horizon Flex, Parallels Desktop.
 
 
 \subsection{DaaS}
 Desktop as a service je variace na centralizovanou virtualizaci. Výpočetní středisko však není uvnitř společnost, ale vlastní jej externí subjekt. Ten pak jednotlivé pracovní stanice pronajímá formou pravidelných poplatků za službu. Pro IT oddělení tak odpadají náklady na správu serverů, síťové infrastruktury a dalších souvisejících opatření. Tento model není pro zkoumanou společnost vhodný, jelikož firemní data by se nacházela mimo společnost a docházelo by tak k ohrožení firemních aktiv.
 
 Nejznámější poskytovatelé jsou: VMWare Horizon Air, Citrix XenDesktop, Amazon Work Spaces.
 
  \subsubsection{Virtualizace aplikací}
 Další možností je nevirtualizovat celý operační systém, ale pouze aplikace. Mezi výhody patří snadná aktualizace aplikací, snadná správa přístupu k aplikacím či nenáročnost na výpočetní výkon klienta. Data a přístupy je takto však možné oddělit pouze v rámci takto nasazených aplikací.
 
 Mezi hlavní nevýhody patří problémy s periferiemi jako např. tiskárny či nutnost stálé a kvalitní konektivity.
 
  Nejznámějšími zástupci služeb pro virtualizace aplikací jsou: Citrix XenApp, VMware Horizon, Dell vWorkspace, and Microsoft RDSH.
  
  
  \subsubsection{Používání webových aplikací}
  Variací na virtualizaci aplikací je jejich úprava pro přístup z webového prohlížeče. To však není možné u všech používaných aplikací ve zkoumané společnosti.
  
 \subsection{Rozlišení na úrovni sítě}
 Použitím Network Access Controll neboli NAC je možné spravovat přístup zařízení do sítě. Je možné nastavit autentifikační kontroly a další bezpečnostní politiky, které musí zařízení splňovat aby bylo do sítě vpuštěno. 
 
 Společnost Gartner identifikuje několik funkcí, které tato řešení nabízejí \cite{GartnerNAC}. Politiky mohou pojmout různé fukce jako například autentifikaci zařízení, autorizaci uživatele, lokaci, čas či přístup k aplikacím a zdrojům.
 
 Dále je možné posoudit stav zařízení z hlediska aktuálnosti systému a antivirových definic co se týče zařízení s Windows, nebo přítomnost EMM, viz \ref{EMM},  na mobilních zařízení.
 
 Přístup je přidělován pomocí síťové infrastruktury s použitím 802.1X protokolu, virtuálních LAN, či seznamů pro řízení přístupů neboli ACL (Access Control List).
 
 Dalšími službami poskytovanými NAC řešeními může být vytváření sítí pro hosty, monitoring připojených zařízení či integrace s dalšími bezpečnostními prvky.
 
 
Gartner ve své studii \cite{GartnerNAC} zmiňuje následující prudukty: Aruba ClearPass, Auconet BICS, Brandford Networks Network Sentry, Cisco ISE, Extreme Networks ExtremeControll, ForeScout CounterACT, Impulse Point SafeConnect, Info Express CGX, Portnox CLEAR, Pulse Policy Secure, SnoopWall Netshield.
 
 

 
 
 \subsubsection{EMM/EMS}\label{EMM}
 Enterprise mobility management nebo též Enterprise Mobility suite umožňují integrovat a spravovat mobilní zařízení v rámci firemní infrastruktury.
 Dle agentury Gartner jsou EMM balíky lepidlem, které připojuje mobilní zařízení do firemní infrastruktury. \cite{Gartner_EMM_2016}
 
 EMM mají následující funkce:
 \begin{itemize}
     \item nastavují zařízení a aplikace pro nasazení ve firemním prostředí
     \item sledují dodržení firemních politik a spravují firemní aktiva
     \item snižují riziko ztráty dat, krádeže či dalších incidentů řízením šifrování dat, přístupových práv, sdílených zařízení, obalováním aplikací či zamknutím zařízení.
     \item umožňují vzdálenou podporu zařízení pro IT oddělení
 \end{itemize}
 
Výzkum společnosti J Gold Associates z roku 2016 \cite{JBBrief} 
ukazuje, že firmy v drtivé většině nenasazují všechny funkce, které EMM řešení nabízejí. Nejpoužívanější funkce EMM jsou vypsány na obrázku (\ref{funkceEMM}) 

\begin{figure}[h!]
\centering
\includegraphics[width=13cm]{img/funkceEMM}
\caption{Které komponenty EMM řešení organizace zapojené do průzkumu aktuáně používají? Převzato z \cite{JBBrief}.} 
\end{figure}\label{funkceEMM}

 
 
 
  \subsection{MDM}
 Mobile device management je software pro správu mobilního zařízení. Je podmnožinou EMM. Mezi základní funkce tohoto software podle \cite{systemOnline} patří:
 \begin{itemize}
     \item Automatické nastavení mobilního zařízení. Umožňuje IT oddělení nastavit zařízení podle firemních potřeb. To zahrnuje instalaci bezpečnostních certifikátů, nastavení uživatelských účtů či dalších nastavení umožňující přístup k firemní síti.
     \item Možnost vzdáleného vymazání. Umožňuje vzdáleně vymazat data tak, aby nebyla dostupné. To je užitečné v případě ztráty či krádeže zařízení, nebo po ukončení pracovního poměru se zaměstnancem.
     \item Vynucení bezpečnostních politik. To zahrnuje vynucení silného hesla, šifrování dat či omezení některých funkcí, například propojení se soukromým cloudovým úložištěm. 
     \item Detekce jailbreak/root zařízení. Detekuje spuštění zařízení v administrátorském režimu, což je ve firemním prostředí nepřípustné.
     \item Blacklisting/whitelisting aplikací. Umožňuje správci zařízení zvolit, které aplikace je a není možné instalovat.
     \item Monitoring. Umožňuje sledovat přístupy uživatele k jednotlivým službám.
     \item Administrace. Umožňuje hromadné aktualizace, instalace či odinstalace aplikací pro zařízení ve firemní flotile.
 \end{itemize}
 

 
 \subsection{MAM} 
 Mobile application management. MAM je též podmnožinou EMM. Narozdíl od MDM nespravuje zařízení jako celek, ale pouze podnikové aplikace. Tyto aplikace jsou získávány přes speciální obchod s aplikacemi. Podle \cite{MAMcitace} mezi hlavní funkce MAM patří:
 \begin{itemize}
     \item Podnikový obchod s aplikacemi. Umožňuje nasazování vlastních i komerčních aplikací pro potřeby businessu.
     \item Podpora správy a distribuce aplikací s užitím API operačního systému či hromadného nákupu aplikací.
     \item Kontejnerizace aplikací
     \item Reporting o užívání aplikací
 \end{itemize}
 
 Podle \cite{Gartner_EMM_2016} jsou pomocí MAM běžně uplatňovány následující politiky:
 \begin{itemize}
     \item Vyžadování iniciace VPN spojení pro aplikaci při spuštění
     \item Šifrování podnikových dat (často s použitím silnějšího šifrování než by bylo použito v rámci operačního systému)
     \item Omezení sdílení dat mezi aplikacemi pouze na podnikové aplikace
     \item Omezení copy/paste funkcionality
     \item Vyžadování specifického stavu při spuštění nebo při přístupu -- například nebyl detekován root nebo jailbreak
 \end{itemize}
 
  
%\begin{figure}[h]
%\centering
%\includegraphics[width=7cm]{img/MAM-Offering}
%\caption{Gartner Magic quadrant. Převzato z } 
%\label{MAM:nacrt}
%\end{figure}\todo{ citace https://www.ibm.com/developerworks/community/blogs/mobileblog/entry/got\_mam\_mobile\_application\_management\_in\_your\_2013\_mobile\_menu25?lang=en}


\subsection{MCM} 
Mobile content management. Jedná se o software pro správu obsahu na mobilních zařízeních. podle \cite{Gartner_EMM_2016} má tři základní role:
\begin{itemize}
    \item Vynucování politik. Dokáže vynutit politiky pro jednotlivé soubory včetně šifrovacích klíčů nezávislých na zařízení, autentifikace, pravidel pro sdílení souborů či pravidel pro copy and paste funkcionalitu.
    \item Přístup k obsahu. Vynutí pravidla pro distribuci, záměnu a mazání souborů.
    \item Integrace Přidává kompaktibilitu pro systémy správy práv, jako jsou ochrana ztráty dat (DLP) nebo podniková správa práv (EDRM) od třetích stran.
\end{itemize}
 
 
 %Vyberte nejvhodnější variantu na trhu dostupného řešení a konfrontujte ji s praxí ve vybrané organizaci. 
\section{Výběr nejvhodnější varianty}

Předchozí analýzy prokázaly, že neexistuje řešení, které by dokázalo zastřešit všechny případy užití vlastních zařízení ve firemním prostředí. Proto tato práce bude dále dělit BYOD podle typu zařízení, a to na mobilní zařízení jako jsou mobilní telefony či tablety a notebooky.

\section{Výběr řešení pro mobilní telefony a tablety}

%\todo{Proc je potreba EMM? Viz specifikace projektu Good}
Podle analýzy \ref{identifikovanePotreby} uživatelé žádají ze svých osobních mobilních telefonů a tabletů přístup k emailům a ke kalendáři. Firma se naopak snaží oddělit firemní data od soukromých tak, aby nad nimi měla kontrolu. Tyto požadavky splňují řešení EMM.  

Trh s nástroji v posledních letech výrazně rostl, zároveň se však konsolidoval \cite{IDC2}. V grafech \ref{EMM:podil2015} a \ref{EMM:podil2016} je patrný nárůst trhu s EMM mezi lety 2014 a 2015 z 1,4 miliardy dolarů na 1,8 miliardy dolarů, tedy o 26,9 \%. Zároveň je vidět zvyšování tržního podílu velkých hráču. Výrazný vliv měla také akvizice společnosti Good Technology společností BlackBerry.

 
\begin{figure}[h!]
\includegraphics[width=13cm]{img/IDC_EMM}
\caption{Podíl jednotlivých poskytovatelů EMM na trhu v roce 2014 podle IDC. Převzato z \cite{IDC1}.} 
\label{EMM:podil2015}
\centering
\end{figure}

\begin{figure}[h!]
\includegraphics[width=13cm]{img/IDC2016}
\caption{Podíl jednotlivých poskytovatelů EMM na trhu v roce 2015 podle IDC. Převzato z \cite{IDC0}.} 
\label{EMM:podil2016}
\centering
\end{figure}

Podle magazínu CIOReview \cite{CIOReview} bylo v roce 2016 pro BYOD nejslibnějších následujících dvacet poskytovatelů softwaru: Accelion, API Systems, Cyber adAPT, Ericom Software, Excelerate Systems, GSG Telco, High Point Solutions, LANDESK Software,  Mathe, MobileIron, MobilityLab, Movius, RES Software, Sirama Consulting, Skycure, Storgrid, Tangoe, Tyfone, VmWare AirWatch, Zix Corporation.

Některé z nich jsou však příliš úzce zaměřené, či jsou pouze minoritními hráči na trhu. Analýza společnosti Gartner \cite{Gartner_EMM_2016} z roku 2016 pro EMM rozděluje jednotlivé poskytovatele dle jejich postavení na trhu a zároveň hodnotí jejich schopnost zohlednit v produktu aktuální požadavky trhu a nasměrování produktu k budoucím potřebám zákazníků. Tato kritéria shrnuje společnost Gartner jako osy "schopnost vykonat" a "úplnost vize" ve svém grafu nazývaném magic quadrant \ref{EMM:quadrant}.



\begin{figure}[h!]
\includegraphics[width=13cm]{img/Gartner_EMM}
\caption{Gartner Magic quadrant. Převzato z \cite{Gartner_EMM_2016}} 
\label{EMM:quadrant}
\centering
\end{figure}
 
Následující společnosti se nacházejí v kvadrantu lídrů:


\subsubsection{VMWare Airwatch}
VMWare koupil společnost AirWatch v roce 2014, viz \cite{VmBuyAir}. Od té doby VMWare zařadil tento EMM do svého porfolia a postupně jej integruje s dalšími produkty jako jsou jeho nástroje pro IAM (Identity and Access Management) a SDN (software-defined networking). AirWatch nabízí širokou podporu pro nástroje třetích stran a je jedním ze zakládajících členů standardu AppConfig. VMWare AirWatch je vhodný pro společnosti, které hledají rozsáhlou funkcionalitu s podporou mnoha platforem.

Podle \cite{Gartner_EMM_2016} byla prokázána nasaditelnost do rozsáhlých prostředí a snadná administrace. Na druhou stranu se objevily problémy s technickou podporou a také nutnost použít řešení od třetí strany pro PIM (Person information management).



\subsubsection{MobileIron}
MobileIron je veřejně obchodovatelná společnost (NASDAQ: MOBL), která se jako jedna z posledních soustředí pouze na svůj EMM produkt. Nabízí však širokou podporu aplikací třetích stran a je jedním ze zakládajících členů standardu AppConfig. Společnost je ceněna pro schopnost přinášet nové funkce na všechny tři hlavní mobilní platformy a plnění amerických bezpečnostních certifikací. Jedná se o produkt, který nabízí mnoho funkcí, škálovatelnost, stabilitu a integraci s dalšími aplikacemi.

Řešení nabízí nástroj pro reporting, pokročilou integraci se SIEM (security information and event management) řešeními třetích stran či správu z mobilního zařízení. Získává kladné ohlasy na svou stabilitu, použitelnost, škálovatelnost a rozsáhlý ekosystém přidružených aplikací AppConnect. MobileIron se drží mezi prvními při nasazování pro nové verze operačních systému.

Na druhou stranu podle \cite{Gartner_EMM_2016} jsou známé případy, kdy zákazníci měli potíže získat technickou podporu. Aplikace Apps@Work nabízejí zastaralý uživatelský zážitek a zároveň existuje nejistota ohledně budoucnosti firmy vzhledem ke změnám ve vrcholném managementu.

\subsubsection{Citrix}
Řešení od společnosti Citrix se skládá z produktů NetScaler, ShareFile a Xen Mobile. Je silné především díky balíku kontejnerizovaných aplikací Worx. ShareFile je kvalitní EFSS (Enterprise file synchronization and sharing) řešení. Obsahuje též uživatelsky přívětivé DLP (Data loss prevention). XenMobile je vhodný pro společnosti s existující infrastrukturou od Citrixu nebo pro ty, jež požadují široké spektrum funkcí.

Společnost Gartner zaznamenala problémy u nasazení XenMobile jako SaaS (Software as a service) u velkých projektů (tj. více než 20000 zařízení) \cite{Gartner_EMM_2016}. Přestože XenMobile nabízí možnost virtualizace Windows aplikací pro mobilní zařízení, použitelnost je na na mobilních zařízeních sporná, vzhledem k dotykové povaze ovládání uživatelského rozhraní. 


\subsubsection{IBM}
IBM nabízí kompletní balík EMM nástrojů MaaS360. Podporuje všechny významné operační systémy, nabízí dobrou spolupráci s dalším bezpečnostním software od IBM. Jedná se o produkt, který má velký záběr, co se týče funkcionality, ale přitom je snadno nasaditelný, viz \cite{Gartner_EMM_2016}.



\subsubsection{BlackBerry}
BlackBerry nyní prodává svůj nástroj jako Good Secure EMM Suite. Skládá se z BES12, Good collaboration apps, Good dynamics a WatchDox Enterprise. Produkty pod značkou Good a WatchDox získala Blacberry akvizicemi které byly dokončeny v roce 2015, viz \cite{BBBuyDox, BBBuyGood}.

Podle agentury Gartner je Good Secure EMM Suite vhodný pro organizace s přísnými požadavky na bezpečnost či působící v regulovaném sektoru. Těm nabízí silnou sadu nástrojů pro ochranu. Zároveň existuje silná podpora pro starší verze software od BlackBerry. Nástroj Good Work nabízí jeden z nejlepších zabezbečených Personal information manager (PIM) nástrojů. Podpora od BlackBerry získává mnoho kladných hodnocení od zákazníků. 

Vícevrstvá cloudová verze produktu BES12 umisťuje data do datacenter ve dvou lokacích, a to Kanadě a Nizozemsku. To by mohl být pro některé bezpečnostní politiky problém. Zároveň u balíku od společnosti BlackBerry dochází k roztříštěnosti služeb mezi jednotlivými produkty.


Další řešení:

\subsubsection{Cisco}
Cisco se dostalo mezi společnosti nabízející MDM software akvizicí společnosti Meraki v roce 2012 \cite{CiscoBuyMeraki}. Kromě řešení pro Android a iOS nabízí také podporu pro Windows a MAC OS X. Nabízí hlubokou integraci do síťové ingrastruktury. Správa produktu nabízí velice jednoduché a přívětivé uživatelské rozhraní. Cenově se jedná o levnější řešení než u většiny konkurentů.

Výhody integrace do síťové infrastruktury je možné využít pouze v případě, že organizace používá sítovou infrastrukturu od Cisco/Meraki. Meraki neobsahuje všechny součásti EMM, soutředí se pouze na MDM.



\subsubsection{Microsoft}
EMM produkt od Microsoftu se nazývá Enterprise Mobility Suite. Skládá se z Microsoft Intune, Azure Active Directory Premium, Advanced Threat Analytics a Azure Rights Management. MDM a MAM služby jsou soustředěny v Microsoft Intune. Toto řešení je nabízeno pouze jako služba v cloudu. Řešení od Microsoftu je vhodné pro společnosti, které nemají vysoké nároky na správu a používají Office 365 nebo Azure Active Directory.

\subsubsection{Landesk}
Landesk se zaměřuje především na UEM (User Evironment Management) a jeho řešení Landesk Mobility Suite tak zapadá do jeho portfolia jako doplněk pro mobilní zařízení. Je tedy vhodné především pro firmy, které mají potřebu spravovat desktopové prostředí a mobilní zařízení zároveň.

\subsubsection{Další řešení}
Ostatní řešení byla v \cite{Gartner_EMM_2016} prezentována jako nabízející příliš úzké zaměření, nedostatečnou funkcionalitu nebo nevhodnost pro nasazení ve větším měřítku


\section{Výběr řešení pro notebooky}

Vzhledem k požadavkům na bezpečnost a dodržování přísných firemních politik v bance a zároveň k potřebě přístupů k různým typům software a aplikací se zdá jako jediné vhodné řešení BYOD virtualizace desktopu. Způsobů, jakými může virtualizace sloužit pro řešení BYOD je více.

Analýza \cite{ForresterWave} ze září roku 2015 od společnosti Forrester se zaměřuje na virtuální desktopy umístěné na vlastním serveru. Výhodou oproti DaaS řešení je, že aplikace i data jsou pod úplnou kontrolou IT oddělení, což snižuje riziko ztráty nebo krádeže dat. Nevýhodou těchto řešení může být problémová funkčnost některých periferních zařízení, jako jsou webové kamery nebo tiskárny. Dále jsou tato řešení velmi citlivá na stabilitu a rychlost internetového připojení, a především u graficky náročnějších aplikací. 

\begin{figure}[h!]
\includegraphics[width=13cm]{img/Forrester_Wave}
\caption{The Forrester Wave: Virtuání desktopy umístěné na serveru. Převzato z: \cite{ForresterWave}.} 
\label{Forrester_Wave}
\centering
\end{figure}

Podle této analýzy jsou jasnými lídry trhu s centralizovanou virtualizací společnosti Citrix a VMWare, a to s obrovským tržním i technologickým náskokem. V grafu \ref{Forrester_Wave} je vidět také společnost Dell, která však již vlastní řešení dále nenabízí a prohlubuje spolupráci s produkty od VMWare, jelikož tuto společnost získala akvizicí jejího původního vlastníka společnosti EMC, v září roku 2016 \cite{DellBuyEMC}.

Průzkum trhu od společnosti IDC \cite{IDCVCC} z roku 2016 má širší zaměření, a to na poskytovatele VCC (Virtual Client Computing). Ty definuje jako poskytovatele, kteří tvoří a prodávají software pro virtualizaci se zaměřením na centralizované virtuální desktopy, distribuované virtuální desktopy a software pro virtuální uživatelské sezení (VUS). Průzkum je zaměřen především na obchodní úspěch hodnocených společností. Dále doporučuje zohlednit při výběru poskytovatele kvalitu systému pro správu zařízení, bezpečnost řešení, možnosti grafického výstupu a kompaktibilitu s užívanými aplikacemi. 

\begin{figure}[h!]
\includegraphics[width=13cm]{img/IDC_VM}
\caption{IDC MarketScape: Hodnocení dodavatelů VCC. Převzato z: \cite{IDCVCC}.} 
\label{IDC_VM}
\centering
\end{figure}

Podle tohoto průzkumu trhu s VCC jasně vládnou společnosti VMWare a Citrix. Nikdo další již nebyl zařazen do segmentu lídrů. Za zmínku dále stojí Microsoft, který má na trhu silnou pozici.



\subsection{Citrix}
Řešení XenDesktop se vyznačuje podporou vlastního protokolu HDX  díky kterému se snaží o adaptivní kompresi, de-duplikaci síťového provozu a přesměrování tíhy renderování dle okolností na klienta a to na všech podporovaných platformách \cite{CitrixHDX}. Dále podporuje vícenásobné 4k monitory a pokročilé funkce pro multimedia a videokonference. Výhodou je podpora amerického bezpečnostního standardu FIPS 140-2.
Podle \cite{ForresterWave} má XenDesktop výborné uživatelské hodnocení, avšak technická podpora je pomalá.

Oproti konkurenčnímu produktu od VMWare nabízí Citrix i virtualizaci Linuxových desktopů. Chlubí se třikrát rychlejším tiskem, šestkrát rychlejším spouštěním aplikací, pětkrát rychlejším ukládání souborů či podporou virtualizovaného Skype for Bussiness. Je možné jej nasadit na jakýkoliv cloud, jakýkoliv hypervizor, síť, do cloudu, lokálně či hybridně, viz \cite{CitrixInfo, CitrixPaper}.

XenDesktop je možné provozovat také v cloudu. Zvolit lze libovolný hypervizor z nabídky VMWare ESX, Microsoft Hyper-V nebo Citrix XenServer. Pro offline použití existuje hypervizor typu 2 pro MacOS a Windows jménem DesktopPlayer. 

Pro virtualizaci aplikací nabízí Citrix platformu XenApp.


\subsection{VMWare}
VMWare nabízí produkt VMWare \textbf{Horizon View}. Použitý protokol je PCoIP od firmy Teradici. Je vhodný v kombinaci serverem vSphere, kdy nabízí dobrou integraci. Nabízí též škálování do cloudu v kooperaci s řešením Horizon Air. Taktéž moduly software od VMWare splňují bezpečnostní standard FIPS 140-2 \cite{VMFIPS}. Horizon View je také možné zakoupit jako součást kompletního balíku, který obsahuje taktéž Horizon Flex pro offline použití. Podle \cite{ForresterWave} hodnotí zákazníci produkt jako dobrý s několika problémy, jako například nutnost použití příkazové řádky pro některá nastavení.

Další produkty pro virtualizaci pracovních prostředí jsou podle výrobce \cite{VMProdukty} následující:


\textbf{Horizon 7} je platforma od VMWare pro virtuální desktopy a aplikace. \textit{Řešení Horizon 7 umožňuje zajišťovat, spravovat a chránit virtuální desktopy (VDI) a aplikace prostřednictvím jedné platformy.}

\textbf{Horizon Air} je DaaS řešení od VMWare. \textit{Poskytuje virtuální desktopy a aplikace hostované v cloudu s širokou škálou možností včetně sdílených desktopů a aplikací.}

\textbf{Horizon Flex} je řešení, které \textit{doručuje, spravuje a zabezpečuje místní virtuální desktopy se systémem Windows na počítačích Mac i PC a současně zajišťuje zabezpečení, možnosti řízení a dodržování požadavků.}

\textbf{App Volumes} je \textit{portfolio integrovaných řešení pro správu aplikací a uživatelů pro virtuální prostředí řešení Horizon, Citrix XenApp a XenDesktop a RDSH.}

\textbf{Mirage} \textit{nabízí správu bitových kopií desktopů pro fyzické desktopy a zařízení POS v nejrůznějších distribuovaných prostředích.}

\textbf{NSX for Horizon} \textit{je síťové řešení infrastruktury virtuálních desktopů (VDI) se zásadami, které jsou dynamicky spojeny s desktopy.}

\textbf{Virtual SAN for Horizon} \textit{Řešení VMware vSAN snižuje zákazníkům počáteční náklady a umožňuje jim využívat celou řadu předkonfigurovaných zařízení pro řešení Horizon, včetně zařízení Virtual SAN Ready Node a infrastruktury postavené na řešení EVO SDDC.}

\textit{VMware \textbf{ThinApp} je řešení pro virtualizaci aplikací bez agentů, které izoluje aplikace od použitých operačních systémů a díky tomu eliminuje konflikty a zjednodušuje doručování a správu.}

\textbf{Řešení User Environment Manager} \textit{nabízí podnikovou správu uživatelů vytvářející přizpůsobené prostředí pro koncové uživatele na všech zařízeních a místech.}

\textbf{Produkty Fusion počítače Mac} \textit{Pomocí řešení VMware Fusion a VMware Fusion je možné používat na počítači Mac bez restartování systém Windows a stovky dalších operačních systémů.}

\textbf{Produkty Workstation systém Windows} Řešení VMware Workstation a VMware Workstation Player představují oborový standard pro používání více operačních systémů jako virtuálních strojů na jednom počítači PC.

\textbf{Řešení Workstation systém Linux} Produkty řešení VMware Workstation systém Linux představují oborový standard pro používání více operačních systémů jako virtuálních strojů na jednom počítači se systémem Linux.

VMware tvrdí, že jeho řešení nabízí oproti konkurenčnímu Citrixu lepší správu a reporting nebo také centrální správu obrazů systémů ať už pro fyzické, virtuální nebo BYOD stroje \cite{VMBetter}. 


\subsection{Microsoft}
Microsoft nabízí virtuální pracovní stanice skrze platformu Windows Server. Nenabízí sice DaaS řešení, ale nabízí virtualizaci aplikací Microsoft Azure RemoteApp. Ty mohou fungovat buďto v čistě cloudovém nebo hybridním módu. VDI je provozováno pod značkou RDS (Remote Desktop Service) jako uživatelské sezení na Windows Server. Z toho důvodu nenabízí tolik možností nastavení a správy jako plná virtualizace \cite{VMwareMicrosoft}.

\subsection{Oracle}
Oracle nabízí nástroj Secure Global Desktop, který je možné použít s různými hypervizory, je ovšem optimalizovaný pro Oracle. Hlavní devízou řešení je kvalitní konzole pro správu Oracle Enterprise Manager. Podle \cite{ForresterWave} toto řešení není vhodné pro případy užití mimo prostředí s vysokým podílem aplikací od Oracle.

Dále nabízí program VirtualBox. Jedná se o hypervizor typu 2, v základní verzi je zdarma i pro komerční užití. Je zaměřený spíše na vývojáře a nenabízí mnoho nástrojů pro vzdálenou správu \cite{OracleVB}.

\subsubsection{Amazon}

Amazon nabízí DaaS službu Amazon Workspaces. Je postavená na platformě Windows server 2008 a používá protokol PCoIP, viz \cite{AmazonAWS}. Je možné volit z mnoha hardwarových konfigurací. Službu lze propojit s firemním Active directory.

\chapter{Návrh řešení}\label{k4}
  %Vyberte nejvhodnější variantu na trhu dostupného řešení a konfrontujte ji s praxí ve vybrané organizaci. 
\section{Výběr nejvhodnější varianty}

Předchozí analýzy prokázaly, že neexistuje řešení, které by dokázalo zastřešit všechny případy užití vlastních zařízení ve firemním prostředí. Proto tato práce bude dále dělit BYOD podle typu zařízení a to na mobilní zařízení jako jsou mobilní telefony či tablety a notebooky.

\section{Výběr řešení pro mobilní telefony a tablety}

\todo{Proc je potreba EMM? Viz specifikace projektu Good}

Trh nástroji v posledních letech výrazně rostl, zároveň se však konsilidoval. \todo{citace https://theictscoop.com/airwatch-consolidates-emm-leadership-in-latest-idc-report-6622602febde   } V grafech \ref{EMM:podil2015} a \ref{EMM:podil2016} je vidět nárůst trhu s EMM mezi lety 2014 a 2015 z 1,4 miliardy dolarů na 1,8 miliardy dolarů, tedy o 26,9 \%. Zároveň je vidět zvyšování tržního podílu velkých hráču. Výrazný vliv měla také akvice společosti Good Technology společností BlackBerry.

 
  \begin{figure}[h]
\includegraphics[width=13cm]{img/IDC_EMM}
\caption{Podíl na trhu jednotlivých poskytovatelů EMM v roce 2014 podle IDC Převzato z \cite{}} 
\label{EMM:podil2015}
\centering
\end{figure}\todo{citace http://www.idc.com/getdoc.jsp?containerId=US40430516  }

  \begin{figure}[h]
\includegraphics[width=13cm]{img/IDC_EMM_2016}
\caption{Podíl na trhu jednotlivých poskytovatelů EMM v roce 2015 podle IDC Převzato z \cite{}} 
\label{EMM:podil2016}
\centering
\end{figure}\todo{citace https://theictscoop.com/airwatch-consolidates-emm-leadership-in-latest-idc-report-6622602febde   }

Podle magazínu CIOReview bylo v roce 2016 pro BYOD nejslibnějších následujících dvacet poskytovatelů software: Accelion, API Systems, Cyber adAPT, Ericom Software, Excelerate Systems, GSG Telco, High Point Solutions, LANDESK Software,  Mathe, MobileIron, MobilityLab, Movius, RES Software, Sirama Consulting, Skycure, Storgrid, Tangoe, Tyfone, VmWare AirWatch, Zix Corporation.

Některé z nich jsou však příliš úzce zaměřené, či jsou pouze minoritními hráči na trhu. Analýza společnosti Gartner \cite{Gartner_EMM_2016} z roku 2016 pro EMM rozděluje jednotlivé poskytovatele dle jejich postavení na trhu a zároveň zohledňuje jejich schopnost zohlednit v produktu aktuální požadavky trhu a nasměrování produktu k budoucím potřebám zákazníků. Tyto kritéria shrnuje společnost Gartner jako osy "schopnost vykonat" a "úplnost vize" ve svém grafu nazývaném magic quadrant. \ref{EMM:quadrant}



 \begin{figure}[h]
\includegraphics[width=13cm]{img/Gartner_EMM}
\caption{Gartner Magic quadrant. Převzato z \cite{Gartner_EMM_2016}} 
\label{EMM:quadrant}
\centering
\end{figure}\todo{citace}
 %\missingfigure{magic quarter}
 
Následující společnosti se nacházejí v kvadrantu lídrů:


\subsubsection{VMWare Airwatch}
VMWare koupil společnost AirWatch v roce 2014 \todo{citace http://ir.vmware.com/overview/press-releases/press-release-details/2014/VMware-Completes-Acquisition-of-AirWatch/default.aspx}. Od té doby VMWare zařadil tento EMM do svého porfolia a postupně jej integruje s dalšími produkty jakou jsou jeho nástroje pro IAM (Identity and Access Management) a SDN (software-defined networking). AirWatch nabízí širokou podporu pro nástroje třetích stran a je jeden ze zakládajících členů standardu AppConfig. VMWare AirWatch je vhodný pro společnosti, které hledají rozsáhlou funkcionalitu s podporou mnoha platforem.

Podle \todo{citace} byla prokázána nasaditelnost do rozsáhlých prostředí a snadná administrace. Na druhou stranu se objevily problémy s technickou podporou a také nutnost použít řešení od třetí strany pro PIM (Person information management).



\subsubsection{MobileIron}
MobileIron je veřejně obchodovatelná společnost (NASDAQ: MOBL), která jako jedna z posledních soustředí pouze na svůj EMM produkt. Nabízí však širokou podporu aplikací třetích stran a je jedním ze zakládajících členů standardu AppConfig. Společnost je ceněna za schopnost přinášet nové funkce na všechny tři hlavní mobilní platformy a plnění amerických bezpečnostních certifikací. Jedná se o produkt, který nabízí mnoho funkcí, , škálovatelnost, stabilitu a integraci s dalšími aplikacemi.

Řešení nabízí vytváření nástroj pro reporting, pokročilou integraci se SIEM (security information and event management) řešeními třetích stran či správu z mobilního zařízení. Získává kladné ohlasy na svou stabilitu, použitelnost, škálovatelnost a rozsáhlý ekosystém přidružených aplikací AppConnect. MobileIron se drží mezi prvními při nasazování pro nové verze operačních systému.

Na druhou stranu podle \todo{citace} jsou známé případy, kdy zákazníci měli obtíže získat technickou podporu, aplikace Apps@Work nabízejí zastaralý uživatelský zážitek a zároveň existuje nejistota ohledně budoucnosti firmy vzhledem ke změnám ve vrcholém managementu.

\subsubsection{Citrix}
Řešení od společnosti Citrix se skládá z produktů NetScaler, ShareFile a Xen Mobile. Je silné především díky balíku kontejnerizovaných aplikací Worx. ShareFile je kvalitní EFSS (Enterprise file synchronization and sharing) řešení. Obsahuje též uživatelsky přívětivé DLP (Data loss prevention).XenMobile je vhodný pro společnosti s existující infrastrukturou od Citrixu nebo pro ty co požadují široké spektrum funkcí.

Společnost Gartner zaznamenala probléby u nasazení XenMobile jako SaaS (Software as a service) u velkých projektů (tj. více než 20000 zařízení). Přestože XenMobile nabízí možnost virtualizace Windows aplikací pro mobilní zařízení, použitelnost je na na mobilních zařízeních sporná, vzhledem k dotykové povaze ovládání uživatelského rozhraní. 


\subsubsection{IBM}
IBM nabízí komletní malík EMM nástrojů MaaS360. Podporuje všechny významné operační systémy, nabízí dobrou spolupráci s dalším bezpečnostním software od IBM. Jedná se o produkt, který má velký záběr co se týče funkcionality, ale přitom je snadno nasaditelný.



\subsubsection{BlackBerry}
BlackBerry nyní prodává svůj nástroj jako Good Secure EMM Suite. Skládá se z BES12, Good collaboration apps, Good dynamics a WatchDox Enterprise. Produkty pod značkou Good a WatchDox získala Blacberry akvizicemi které byly dokončeny v roce 2015. \todo{citace http://global.blackberry.com/en/company/newsroom/press?id=1998017} \todo{http://global.blackberry.com/en/company/newsroom/press?id=1946553}

Podle agentury Gartner je Good Secure EMM Suite vhodný pro organizace s přísnými požadavky na bezpečnost či působící v regulovaném sektoru. Těm nabízí silnou sadu nástrojů pro ochranu. Zároveň existuje silná podpora pro starší verze software od BlackBerry. Nástroj Good Work nabízí jeden z nejlepších zabezbečených Personal information manager (PIM) nástrojů. Podpora od BlackBerry získává mnoho kladných hodnocení od zákazníků. 

Vícevrstvá cloudová verze produktu BES12 umisťuje data do datacenter ve dvou lokacích a to Kanady a Nizozemí. To by mohl být pro některé bezpečnostní politiky problém. Zároveň u balíku od společnosti BlackBerry dochází k roztříštěnosti služeb mezi jednotlivými produkty.

\todo{funkce a tak}

Další řešení:

\subsubsection{Cisco}
Cisco se dostalo mezi společnosti nabízející MDM software akvizicí společnosti Meraki v roce 2012 \todo{citace https://newsroom.cisco.com/press-release-content?articleId=1118649}. Kromě řešení pro Android a iOS nabízí také podporu pro Windows a MAC OS X. Nabízí hlubokou integraci do síťové ingrastruktury. Správa produktu nabízí velice jednoduché a přívětivé uživatelské rozhraní. Cenově se jedná o levnější řešení než většina konkurence.

Výhody integrace do síťové infrastruktury je možné využít pouze v případě, že organizace používá sítovou infrastrukturu od Cisco/Meraki. Meraki neobsahuje všechny součásti EMM, soutředí se pouze na MDM.



\subsubsection{Microsoft}
EMM produkt od Microsoftu se nazývá Enterprise Mobility Suite. Skládá se z Microsoft Intune, Azure Active Directory Premium, Advanced Threat Analytics a Azure Rights Management. MDM a MAM služby jsou soustředěny v Microsoft Intune. Toto řešení je nabízeno pouze služba v cloudu. Řešení od Microsoftu je vhodné pro společnosti, které nemají vysoké nároky na správu a používají Office 365 nebo Azure Active Directory.

\subsubsection{Landesk}
Landesk se zaměřuje především na UEM (User Evironment Management) a jeho řešení Landesk Mobility Suite tak zapadá do jeho portfolia jako doplněk pro mobilní zařízení. Je tedy vhodné především pro firmy, které mají potřebu spravovat desktopové prostředí a mobilní zařízení zároveň.

\subsubsection{Další řešení}
Ostatní řešení byla v \todo{citace} byla prezentována jako mající příliš úzké zaměření, nedostatečnou funkcionalitu nebo nevhodnost pro nasazení ve větším měřítku


\section{Výběr řešení pro notebooky}

Vzhledem k požadavkům na bezpečnost a dodržování přísných firemních politik v bance a zároveň k potřebě přístupů k různým typům software a aplikací se zdá jako jediné vhodné řešení BYOD virtualizace desktopu. Způsobů, jakými může virtualizace sloužit pro řešení BYOD je více.

Analýza \ref{Forreste_Wave} z září roku 2015 od společnosti Forrester se zaměřuje na virtuální desktopy umístěné na vlastním serveru. Výhodou oproti DaaS řešení je, že aplikace i data jsou pod úplnou kontrolou IT oddělení, což snižuje riziko ztráty nebo krádeže dat. Nevýhodou těchto řešení může být problémová funkčnost některých periferních zařízení, jako jsou webové kamery nebo tiskárny. Dále jsou tato řešení velmi citlivá na stabilitu a rychlost internetového připojení a to především u graficky náročnějších aplikací. 

 \begin{figure}[h]\label{Forrester_Wave}
\includegraphics[width=13cm]{img/Forrester_Wave}
\caption{The Forrester Wave: Virtuání desktopy umístěné na serveru} 
\label{EMM:quadrant}
\centering
\end{figure}\todo{citace https://www.citrix.com/content/dam/citrix/en\_us/documents/products-solutions/forrester-wave-server-hosted-virtual-desktops-q3-2015.pdf}

Podle této analýzy jsou jasnými lídry trhu s centralizovanou virtualizací společnosti Citrix a VMWare a to s obrovským tržním i technologickým náskokem. V grafu \ref{Forrester_Wave} je vidět také společnost Dell, která však již vlastní řešení dále nenabízí a prohlubuje spolupráci s produkty od VMWare, jelikož tuto společnost získala akvizicí jejího původního vlastníka, společnosti EMC, v září roku 2016. \todo{citace https://www.emc.com/about/news/press/2016/20160907-01.htm} 

Průzkum trhu od společnosti IDC z roku 2016 \todo{citovat IDC VM} \ref{IDC_VM} má širší zaměření, a to na poskytovatele VCC (Virtual Client Computing). Ty definuje jako poskytovatele kteří tvoří a prodávají software pro virtualizaci se zaměřením na centralizované virtuální desktopy, distribuované virtuální desktopy a software pro virtuální uživatelské sezení (VUS). Průzkum je zaměřen především na obchodní úspěch hodnocených společností. Dále doporučuje zohlednit při výběru poskytovatele kvalitu systému pro správu zařízení, bezpečnost řešení, možnosti grafického výstupu a kompaktibilitu s užívanými aplikacemi. 

 \begin{figure}[h]\label{IDC_VM}
\includegraphics[width=13cm]{img/IDC_VM}
\caption{IDC MarketScape: Hodnocení dodavatelů VCC } 
\label{EMM:quadrant}
\centering
\end{figure}\todo{citace http://campaign.vmware.com/imgs/GlobalCampaigns/39249/IDC\_MarketScape\_Worldwide\_Virtual\_Client\_Computing\_Software\_2016\_Vendor\_Assessment.pdf}

Podle tohoto průzkumu trhu s VCC jasně vládnou společnosti VMWare a Citrix. Nikdo další již nebyl zařazen do segmentu lídrů. Za zmínku dále stojí Microsoft, který má silnou pozici na trhu.



\subsection{Citrix}
Řešení XenDesktop se vyznačuje podporou vlastního protokolu HDX \todo{citace https://www.citrix.com/blogs/2014/08/18/citrix-xendesktopxenapp-what-is-hdx-its-not-just-ica/} díky kterému se snaží o adaptivní kompresi, de-duplikaci síťového provozu a přesměrování tíhy renderování dle okolností na klienta a to na všech podporovaných platformách. Dále podporuje vícenásobné 4k monitory a pokročilé funkce pro multimedia a videokonference. Výhodou je podpora amerického bezpečnostního standardu FIPS 140-2.
Podle \todo{citace forrestera} má XenDesktop výborné uživatelské hodnocení, avšak technická podpora je pomalá.


\subsection{VMWare}
VMWare nabízí produkt VMWare \textbf{Horizon View}. Použitý protokol je PCoIP od firmy Teradici. Je vhodný v kombinaci serverem vSphere, kdy nabízí dobrou integraci. Nabízí též škálování do cloudu v kooperaci s řešením Horizon Air. Taktéž moduly software od VMWare splňují bezpečnostní standard FIPS 140-2. \todo{citace http://www.vmware.com/security/certifications/fips.html}. Horizon View je také možné zakoupit jako součást kompletního balíku, který obsahuje taktéž Horizon Flex pro offline použití. Podle \todo{zase ocitovat forrestera} hodnotí zákazníci produkt jako dobrý s několika problémy, jako například nutnost použití příkazové řádky pro některá nastavení.

Další produkty pro virtualizaci pracovních prostředí jsou podle výrobce \todo{citace http://www.vmware.com/cz/products.html} následující:


\textbf{Horizon 7} je platforma od VMWare pro virtuální desktopy a aplikace. \textit{Řešení Horizon 7 umožňuje zajišťovat, spravovat a chránit virtuální desktopy (VDI) a aplikace prostřednictvím jedné platformy.}

\textbf{Horizon Air} je DaaS řešení od VMWare. \textit{Poskytuje virtuální desktopy a aplikace hostované v cloudu s širokou škálou možností včetně sdílených desktopů a aplikací.}

\textbf{Horizon Flex} je řešení, které \textit{doručuje, spravuje a zabezpečuje místní virtuální desktopy se systémem Windows na počítačích Mac i PC a současně zajišťuje zabezpečení, možnosti řízení a dodržování požadavků.}

\textbf{App Volumes} je \textit{portfolio integrovaných řešení pro správu aplikací a uživatelů pro virtuální prostředí řešení Horizon, Citrix XenApp a XenDesktop a RDSH.}

\textbf{Mirage} \textit{nabízí správu bitových kopií desktopů pro fyzické desktopy a zařízení POS v nejrůznějších distribuovaných prostředích.}

\textbf{NSX for Horizon} \textit{je síťové řešení infrastruktury virtuálních desktopů (VDI) se zásadami, které jsou dynamicky spojeny s desktopy.}

\textbf{Virtual SAN for Horizon} \textit{Řešení VMware vSAN snižuje zákazníkům počáteční náklady a umožňuje jim využívat celou řadu předkonfigurovaných zařízení pro řešení Horizon, včetně zařízení Virtual SAN Ready Node a infrastruktury postavené na řešení EVO SDDC.}

\textit{VMware \textbf{ThinApp} je řešení pro virtualizaci aplikací bez agentů, které izoluje aplikace od použitých operačních systémů a díky tomu eliminuje konflikty a zjednodušuje doručování a správu.}

\textbf{Řešení User Environment Manager} \textit{nabízí podnikovou správu uživatelů vytvářející přizpůsobené prostředí pro koncové uživatele na všech zařízeních a místech.}

\textbf{Produkty Fusion počítače Mac} \textit{Pomocí řešení VMware Fusion a VMware Fusion je možné používat na počítači Mac bez restartování systém Windows a stovky dalších operačních systémů.}

\textbf{Produkty Workstation systém Windows} Řešení VMware Workstation a VMware Workstation Player představují oborový standard pro používání více operačních systémů jako virtuálních strojů na jednom počítači PC.

\textbf{Řešení Workstation systém Linux} Produkty řešení VMware Workstation systém Linux představují oborový standard pro používání více operačních systémů jako virtuálních strojů na jednom počítači se systémem Linux.


\subsection{Microsoft}

\subsection{Oracle}
Oracle nabízí nástroj Secure Global Desktop, který je možné použít s různými hypervizory, je ovšem optimalizovaný pro Oracle. Hlavní devízou řešení je kvalitní konzole pro správu Oracle Enterprise Manager. Podle \todo{zase cituj forestra} toto řešení není vhodné pro případy užití mimo prostředí s vysokým podílem aplikací od Oracle.



%Konzultujte navrhované řešení se zástupci vybrané organizace a stanovte doporučení pro nasazení. 
\section{Hodnocení navrhované varianty zástupci KB}

%Navrhněte nasazení řešení BYOD. 
\section{Návrh řešení}

\todo{vybiram Flex - a proc? http://blogs.gartner.com/mark-lockwood/2014/10/14/is-vmwares-horizon-flex-the-answer-to-byod/}

\todo{Nasazeni - Cisco guide}
\todo{Nasazeni - Cisco Airwatch integration}


%Zhodnoťte uskutečnitelnost řešení a analyzujte benefity a rizika spojená se zavedením navrženého konceptu.
\section{Analýza navrženého řešení}

\chapter{Vyhodnocení}\label{k5}
   %Zhodnoťte uskutečnitelnost řešení a analyzujte benefity a rizika spojená se zavedením navrženého konceptu.
V této kapitole je zhodnocena uskutečnitelnost řešení a analyzovány benefity a rizika spojená se zavedením navrženého konceptu.

\section{Sumarizace navrženého řešení}
Vzhledem k povaze problému, zjištěného analýzou uvnitř společnosti, bylo navrženo zvolit rozdílná řešení podle typu zařízení.

Bylo navrženo postavit koncepci BYOD na třech základních pilířích a to:

\begin{itemize}
    \item Technické řešení pro notebooky
    \item Technické řešení pro mobilní zařízení
    \item Nastavení dalších opatření
\end{itemize}


\section{Uskutečnitelnost navrženého řešení}
%\todo{Zhodnoťte uskutečnitelnost řešení a analyzujte benefity a rizika spojená se zavedením navrženého konceptu}
Z technického hlediska je návrh uskutečnitelný a po odladění technických detailů by z tohoto pohledu nic nebránilo jeho nasazení. Z pohledu obchodního má taktéž smysl, jelikož uskutečněním navržených opatření by se zvýšila bezpečnost informačních a komunikačních systémů zkoumané organizace, přičemž právě vnitřní bezpečnost je vzhledem k jejímu oboru podnikání důležitou součástí korporátní strategie. 

\section{Benefity řešení pro notebooky}

Hlavním benefitem řešení pro BYOD notebooky za použití distribuované virtualizace jsou nízké náklady na nasazení a provoz i vysoká flexibilita. Na rozdíl od centralizované virtualizace s sebou nenese vysoké náklady na úložiště, servery a síťovou infrastrukturu. Stírá známé problémy centralizované virtualizace jako jsou nízká odezva, vysoká citlivost na kvalitu připojení či problémy s provozováním graficky náročných aplikací.

Zároveň však odděluje pracovní prostředí od operačního systému soukromého počítače a zajišťuje tak bezpečnost firemních dat. Velkým benefitem je možnost práce offline při zachování bezpečnosti díky užití virtualizace. Řeší problém s kontraktory a nekontrolovanými zařízeními ve vlastní síti.

Co se týče obecných výhod BYOD, pokud by do programu vstoupilo významné množství zaměstnanců, dá se předpokládat snížení nákladů na IT v rovině snížení nákladů na pořizování firemních zařízení. Dále se dá předpokládat zvýšení spokojenosti u zaměstnanců, kteří by vstoupili do BYOD programu z důvodu nespokojenosti s firemním zařízením. Tyto důvody by však byly pouze vedlejším efektem, hlavním důvodem pro zavedení BYOD programu je nastavení rámce pro existující nefiremní zařízení z důvodu bezpečnosti.

\section{Rizika nasazení navrhovaného řešení pro notebooky}
Největším rizikem nasazení tohoto řešení může být neochota uživatelů přistoupit na tento model, který přináší uživateli nutnost nainstalovat si na své zařízení klientský software a pracovat s ním. U kontraktorů, kteří jsou zaměstnanci partnerských dodavatelských společností, není samozřejmostí povolení virtualizace na jejich pracovních zařízeních a je nutné tuto možnost pro vstup do BYOD programu zajistit.  Dále je třeba individuálně řešit licencování nejrůznějšího softwaru, jelikož není možné v návrhu řešení BYOD programu obsáhnout všechny možné kombinace potřebného software a licenčních politik. Z hlediska bezpečnosti řešení odstraňuje aktuální bezpečnostní hrozby spojené s připojováním nefiremních zařízení. Bezpečnost se tak v tomto ohledu zvyšuje a nebyly identifikovány žádné přidané bezpečnostní hrozby. 

\section{Benefity navrhovaného řešení pro mobilní telefony a tablety}
Hlavní benefit navrženého řešení pro BYOD mobilní telefony a tablety je vůbec možnost využití konceptu BYOD a tím pádem možnost pracovat kdekoliv. To má potenciál zvýšení produktivity a spokojenosti zaměstnanců. Díky zvoleným licencím a typu nasazení uživatele nijak neomezuje v užívání jejich zařízení a nabízí tak vybalancování pracovního a soukromého života. Důvodem k navržení zavedení konkrétního produktu bylo vysoké zaměření na bezpečnost, které je v bankovním prostředí nejvyšší prioritou.

\section{Rizika nasazení navrhovaného řešení pro mobilní telefony a tablety}
Rizikem může být nedůvěra zaměstnanců k firemnímu softwaru, se schopností kontrolovat jejich soukromé zařízení. Reálný provoz také může ukázat u BYOD zařízení snížení výkonu či snížení výdrže na baterii v určitých konfiguracích. To by znamenalo negativní postoj zaměstnanců k BYOD programu a jeho možný neúspěch.

\section{Další dopady realizace projektu}
Procesní změny nutné k zavedení navrhovaného BYOD programu nejsou nikterak závažné a proto nejsou překážkou k realizaci projektu. Díky nastolení programu pro BYOD je možné nabídnout BYOD některým zaměstnancům a považovat to za pracovní benefit.




\begin{conclusion}\label{k6}
	%sem napište závěr Vaší práce
	\missingfigure{Obrázek s kocickou na konec}
\end{conclusion}


\bibliographystyle{csn690}
\bibliography{mybibliographyfile}

\appendix
\chapter{Seznam použitých zkratek}
% \printglossaries
\begin{description}
	%\item[GUI] Graphical user interface
	%\item[XML] Extensible markup language
	\item[BYOD] Bring your own device
\end{description}

\input{sablona_NAVOD}


\chapter{Obsah přiloženého CD}
%upravte podle skutecnosti
 \input{texty/cd}
 
\chapter{HW požadavky VMware Horizon FLEX}\label{pozadavky}

\section{Horizon FLEX Server}
\begin{itemize}
\item Minimum CPU: 1 Quad-Core Processor or 2 vCPUs
\item 2.26GHz Intel core speed or equivalent
\item Minimum 512MB /Recommended 4GB
\item Disk: 10GB+ /Recommended 40GB+
\item Windows 2008 R2, Windows 2012 and above
\item .NET 3.5 SP1 and above
\item IIS 7.0+ with II6 Management Compatibility and ASP.Net
\end{itemize}

\section{VMware Mirage Server}
\begin{itemize}
\item  Minimum 4 vCPU, Recommended 8 vCPU
\item Minimum 8GB RAM, Recommended 16GB
\item  146GB Free Disk Space
\item Windows 2008 R2, Windows 2012 and above
\item .NET 3.5 SP1 and above
\item IIS 7.0+ with II6 Management Compatibility and ASP.Net 
\end{itemize}




\end{document}
